%%%%%%%%%%%%%%%%%%%%%%%%%%%%%%%%%%%%%%%%%%%%%%%%%%%%%%%%%%%%%%%%%%%%%%%%%%%%%%%%
% Template for USENIX papers.
%
% History:
%
% - TEMPLATE for Usenix papers, specifically to meet requirements of
%   USENIX '05. originally a template for producing IEEE-format
%   articles using LaTeX. written by Matthew Ward, CS Department,
%   Worcester Polytechnic Institute. adapted by David Beazley for his
%   excellent SWIG paper in Proceedings, Tcl 96. turned into a
%   smartass generic template by De Clarke, with thanks to both the
%   above pioneers. Use at your own risk. Complaints to /dev/null.
%   Make it two column with no page numbering, default is 10 point.
%
% - Munged by Fred Douglis <douglis@research.att.com> 10/97 to
%   separate the .sty file from the LaTeX source template, so that
%   people can more easily include the .sty file into an existing
%   document. Also changed to more closely follow the style guidelines
%   as represented by the Word sample file.
%
% - Note that since 2010, USENIX does not require endnotes. If you
%   want foot of page notes, don't include the endnotes package in the
%   usepackage command, below.
% - This version uses the latex2e styles, not the very ancient 2.09
%   stuff.
%
% - Updated July 2018: Text block size changed from 6.5" to 7"
%
% - Updated Dec 2018 for ATC'19:
%
%   * Revised text to pass HotCRP's auto-formatting check, with
%     hotcrp.settings.submission_form.body_font_size=10pt, and
%     hotcrp.settings.submission_form.line_height=12pt
%
%   * Switched from \endnote-s to \footnote-s to match Usenix's policy.
%
%   * \section* => \begin{abstract} ... \end{abstract}
%
%   * Make template self-contained in terms of bibtex entires, to allow
%     this file to be compiled. (And changing refs style to 'plain'.)
%
%   * Make template self-contained in terms of figures, to
%     allow this file to be compiled. 
%
%   * Added packages for hyperref, embedding fonts, and improving
%     appearance.
%   
%   * Removed outdated text.
%
%%%%%%%%%%%%%%%%%%%%%%%%%%%%%%%%%%%%%%%%%%%%%%%%%%%%%%%%%%%%%%%%%%%%%%%%%%%%%%%%

\documentclass[letterpaper,twocolumn,10pt]{article}
\usepackage{usenix-2020-09}

% to be able to draw some self-contained figs
% \usepackage{tikz}
% \usepackage{amsmath}

% % inlined bib file
% \usepackage{filecontents}

%%%% %%%% %%%% %%%%
%%%%% Packages and Commands %%%%%

\usepackage[utf8]{inputenc} % allow utf-8 input
\usepackage[T1]{fontenc}    % use 8-bit T1 fonts
%\usepackage{hyperref}       % hyperlinks
\usepackage{url}            % simple URL typesetting
\usepackage{booktabs}       % professional-quality tables
\usepackage{amsfonts}       % blackboard math symbols
\usepackage{nicefrac}       % compact symbols for 1/2, etc.
\usepackage{microtype}      % microtypography
\usepackage{xcolor}         % colors
\usepackage{float}

\usepackage{multirow}
\usepackage{wrapfig}

\usepackage{graphics}
\usepackage{graphicx}
\usepackage{amsmath}
\usepackage{wrapfig}
\usepackage[subfigure]{tocloft}
\usepackage{subcaption}
\usepackage{amsfonts,bm}
\usepackage{natbib} 
\usepackage{makecell}

\usepackage{algorithm}
\usepackage{adjustbox}


\usepackage{algpseudocode}
\usepackage{amsmath}
\usepackage{float}
\usepackage{multicol}
\usepackage{setspace}

\usepackage{tikz}
\usetikzlibrary{shapes.geometric, arrows}
\usepackage{circuitikz}
 \usepackage{tabularray}
 \usepackage{multirow}
\usepackage{colortbl}
%\usepackage{cleveref}
\usepackage{ifthen}
\usepackage{listings}
\usepackage{tabularx}


\usepackage[inline]{enumitem}
\usepackage{hyperref}
\usepackage[capitalise,nameinlink]{cleveref}

%%%% Commands %%%%
% FIXME
\newcommand{\fixme}[1]{\textcolor{red}{#1}}
%CIRCLED
% \DeclareRobustCommand*\circled[1]{\tikz[baseline=(char.base)]{
%           \node[shape=circle,draw,fill=black,text=white,font=\bf,inner sep=1pt] (char)
%             {\normalsize#1};}}

\DeclareRobustCommand*\circled[1]{%
  \tikz[baseline=(char.base)]{%
    \node[shape=circle,
          draw=black, line width=0.5pt,  % black outline at 0.5pt
          fill={rgb,255:red,245;green,245;blue,245}, % F5F5F5
          text=black,
          inner sep=0pt,
          minimum size=10pt] (char) {\fontsize{10pt}{10pt}\selectfont #1};}}
          
% TOPK
\newcommand{\topk}{\texttt{top-k }}
% R+A
\newcommand{\RA}{\texttt{BA}}

%s Section ref
\newcommand{\xref}[1]{\S\ref{#1}}
% Rishabh's macro
%\newcommand\rishabh[1]{\textcolor{blue}{RJ: #1}}
\newboolean{showcomments}
\setboolean{showcomments}{true} % Change to false to hide the text

\newcommand{\rishabh}[1]{%
  \ifthenelse{\boolean{showcomments}}%
    {\textcolor{blue}{RJ: #1}}%  % if true, print normally
    {\phantom{\textcolor{blue}{RJ: #1}}}% % if false, print an invisible copy (space preserved)
}

%\newcommand\design[1]{\textcolor{black}{MaestroRAG}}

\newcommand\todo[1]{\textcolor{red}{TODO: #1}}
% \renewcommand\rishabh[1]{}
% table needs
\usepackage[table]{xcolor} % For cell coloring
\usepackage{makecell} % For multi-line cells
\usepackage{pifont} % For tick and cross marks
\usepackage{booktabs} % For better table styling
\usepackage{comment}

\newcommand{\cmark}{\ding{51}} % Tick mark
\newcommand{\xmark}{\ding{55}} % Cross mark

\newcommand{\flashRAG}{FlashRAG~\cite{jin2024flashrag}}
\newcommand{\edgeRAG}{EdgeRAG~\cite{EdgeRAG}}
\newcommand{\pipeRAG}{PipeRAG~\cite{jiang2024piperag}}

\newcommand{\design}{Prophet } % Change Name here


%\usepackage{amsmath}
\DeclareMathOperator*{\argmax}{arg\,max}
\DeclareMathOperator*{\argmin}{arg\,min}
%%%%%%%%%%%%%%%%%%


%%%% %%%% %%%% %%%%

% Define the circled command
% \newcommand{\circled}[1]{%
%   \tikz[baseline=(char.base)]{
%     \node[shape=circle,
%           fill={rgb,255:red,245;green,245;blue,245}, % F5F5F5 color
%           text=black,
%           inner sep=0pt,
%           minimum size=15pt] (char) {\fontsize{10pt}{10pt}\selectfont #1};}}

%\usepackage{xcolor}
\usepackage{tcolorbox}

% Style for the box
\tcbset{
  colback=blue!5!white,
  colframe=blue!75!black,
  boxrule=0.5pt,
  arc=2pt,
  left=1pt,
  right=1pt,
  top=1pt,
  bottom=1pt
}

% Define counter and environment
\newcounter{insightcounter}
\newenvironment{insight}{
  \refstepcounter{insightcounter}%
  \begin{tcolorbox}
  \textbf{Insight \theinsightcounter:} \em
}{
  \end{tcolorbox}
}


% Python code style
\lstdefinestyle{mypython}{
    language=Python,
    basicstyle=\ttfamily\small,
    keywordstyle=\color{blue}\bfseries,
    commentstyle=\color{green!50!black},
    stringstyle=\color{orange},
    showstringspaces=false,
    breaklines=true,
    frame=single,
    tabsize=4
}
%%%% %%%% %%%% %%%%

%-------------------------------------------------------------------------------
\begin{document}
%-------------------------------------------------------------------------------

%don't want date printed
%\date{}

% make title bold and 14 pt font (Latex default is non-bold, 16 pt)
\title{Prophet: Neural Expert Prediction for Efficient Mixture-of-Experts Inference}

%for single author (just remove % characters)
\author{~
% {\rm Your N.\ Here}\\
% Your Institution
% % \and
% % {\rm Second Name}\\
% % Second Institution
% % copy the following lines to add more authors
% % \and
% % {\rm Name}\\
% %Name Institution
} % end author


\maketitle

%-------------------------------------------------------------------------------
\begin{abstract}
%-------------------------------------------------------------------------------
Mixture-of-Experts (MoE) models enable trillion-parameter language models through sparse activation, but suffer from severe memory bandwidth bottlenecks, where expert loading dominates 78--96\% of the inference time. Current reactive approaches fail to exploit the inherent structure in expert routing patterns, leading to I/O-to-compute ratios of up to $15\times$ that limit practical deployment.

This paper introduces \textit{Prophet}, a neural expert prefetching system that transforms MoE inference through predictive memory management. Prophet employs a lightweight neural predictor to capture cross-layer routing dependencies, combined with batch-aware deduplication and hierarchical caching to optimize expert loading. Operating as a plug-and-play solution without model modifications, Prophet delivers $1.5\times$--$3.2\times$ performance speedups and $1.5\times$--$15\times$ memory improvements over state-of-the-art prefetching across diverse MoE architectures, enabling practical deployment of trillion-parameter models from edge to datacenter environments.

% Mixture-of-Experts (MoE) models enable trillion-parameter language models through sparse activation, but suffer from severe memory bandwidth bottlenecks where
%   expert loading dominates 87--95\% of inference time. Current reactive approaches fail to exploit the inherent structure in expert routing patterns, leading to
%   I/O-to-compute ratios up to $15\times$ that severely limit practical deployment across diverse hardware environments.

%   This paper introduces \textit{Prophet}, a comprehensive neural expert prefetching system that transforms MoE inference through predictive memory management.
%   Through analysis of expert routing traces, we identify exploitable structures including power-law popularity distributions, temporal correlations, and
%   cross-layer dependencies that enable accurate prediction of future expert requirements. Prophet employs a lightweight neural predictor with cross-layer
%   attention to capture routing dependencies across transformer layers, achieving high prediction accuracy with minimal computational overhead.

%   Prophet integrates three synergistic components: neural prediction for proactive expert loading, batch-aware deduplication exploiting sublinear scaling
%   properties for substantial memory savings, and hierarchical caching with confidence-driven prefetching strategies. Operating as a plug-and-play solution
%   without requiring model modifications, Prophet adapts automatically across sparse and dense routing strategies.

%   Comprehensive evaluation demonstrates Prophet's effectiveness: $2\times$--$15\times$ speedups over reactive baselines and $1.8\times$--$4.2\times$ improvements
%    over state-of-the-art prefetching systems across diverse MoE architectures, enabling practical deployment of trillion-parameter models from edge devices to
%   data centers.
\end{abstract}


%-------------------------------------------------------------------------------
%%%%%%%%%% %%%%%%%%%% %%%%%%%%%% %%%%%%%%%%
%%%%%%%%%%  CHAPTERS GOES HERE   %%%%%%%%%%
% Organization:
%src :      Chapters
%figs:      All diagrams
%tabNalgo:  Tables, Algorithms, Pseudocodes
%appendix:  Appendix Chalters

% Start Chapters
%%%%%%%%%% %%%%%%%%%% %%%%%%%%%% %%%%%%%%%%
\section{Introduction}
\label{sec:01Introduction}

% Text matter begin
%%%%%%%%%% %%%%%%%%%% %%%%%%%%%% %%%%%%%%%%

Mixture-of-Experts (MoE) architectures have emerged as the dominant paradigm for scaling language models to unprecedented parameter counts while maintaining computational efficiency~\cite{jiang2024mixtral, gupta2024dbrx,dernbach2024glam, fedus2022switch,costa2022no, flame}. By routing each input token to a small subset of specialized expert networks rather than processing through the entire model~\cite{go2025moetuneroptimizedmixtureexpert}, MoE systems enable trillion-parameter models~\cite{fedus2022switch} to achieve the computational footprint of much smaller dense networks~\cite{cao2025moe,artetxe2022efficientlargescalelanguage}. Despite this promise, MoE models face a fundamental systems challenge that prevents them from realizing their theoretical efficiency gains: expert loading dominates inference time due to severe memory bandwidth bottlenecks~\cite{pregated, hobbit, zhang2025daop,kamahori2025fiddlercpugpuorchestrationfast}.

\noindent\textbf{The MoE Efficiency Paradox.}
The computational efficiency of MoE architectures stems from sparse activation patterns, where each token activates only a fraction of available experts~\cite{zhou2022mixture}. Switch Transformer~\cite{fedus2022switch} employs top-1 routing, activating 1 out of 128 experts per token, while Qwen MoE~\cite{yang2025qwen3, team2024qwen2, bai2023qwen} uses top-8 routing to activate 8 out of 64 experts.
This sparsity enables models to scale to trillions of parameters~\cite{fedus2022switch} while requiring only the computational resources of hundred-billion parameter dense models during inference, representing a breakthrough in efficient scaling.

However, this computational efficiency creates a severe memory paradox. While computation requirements scale with activated experts, memory requirements scale with \textit{total} experts, creating fundamental resource mismatches~\cite{zadouri2023pushing,cao2025moe, moeinfinity}. Switch Transformer~\cite{fedus2022switch} requires approximately 806MB per MoE layer to store all experts, while Qwen3-MoE~\cite{yang2025qwen3} demands 23GB for its 60 routing experts---far exceeding typical GPU memory capacity. More critically, the dynamic nature of expert routing creates unpredictable memory access patterns that defeat traditional caching strategies.


\begin{figure}[t]
\centering
\includegraphics[width=\linewidth]{figs/moe_IO_benchmark_H100.pdf}
\vspace{-10pt}
\caption{I/O-to-compute profiling for Switch and Qwen models across batch sizes shows that host/device I/O and memory movement dominate execution time.}
\label{fig:io_bottleneck}
\vspace{-10pt}
\end{figure}

\noindent\textbf{The Memory Bandwidth Crisis.}
Prior works~\cite{pregated,promoe, expertflow, hobbit, fate} and our empirical analyses, depicted in \Cref{fig:io_bottleneck}, reveal that inference performance is dominated by expert loading rather than computation. While \Cref{fig:io_bottleneck} profiles Switch Transformer and Qwen1.5-MoE (78--96\% I/O), larger architectures exhibit even more severe bottlenecks: our profiling reveals GPT-OSS-20B spends 92.2\% of inference time on expert transfer, Mixtral-8x7B spends 89.3\%, and DeepSeek-MoE-16B spends 74.0\%.
The pronounced I/O--compute imbalance~\cite{zadouri2023pushing, doucet2025harmoeny} leads to GPU underutilization~\cite{zhou2025floe} and makes expert loading the principal determinant of latency across batch sizes and model scales.

\noindent\textbf{Why Prediction is Feasible.}
Effective prefetching requires exploitable structure in expert routing patterns---if routing decisions were truly random, prediction-based prefetching would provide no benefit over simple caching. Our comprehensive information-theoretic analysis (\Cref{sec:Motivation}) establishes that such structure exists and is quantifiable. Expert selection follows robust power-law distributions ($\alpha = 1.2$--$2.2$) across architectures, with the top 20\% of experts handling 75\% of routing decisions. More critically, cross-layer routing exhibits 0.62 bits of mutual information---reducing prediction uncertainty by 54\% compared to random chance. These patterns emerge from model architecture rather than input semantics: when trained on two datasets and evaluated on a third, prediction accuracy drops only 3--8\%, demonstrating that routing structure is model-intrinsic and generalizable. This principled foundation distinguishes Prophet from prior heuristic approaches that lack theoretical justification for prediction feasibility.

\noindent\textbf{Limitations of Current Approaches.}
Existing solutions to the expert loading bottleneck fall into two categories.
\textit{Reactive approaches} employ on-demand expert loading with traditional cache replacement policies~\cite{hobbit}, providing zero overlap between expert loading and computation. \textit{Predictive approaches} attempt to preload experts~\cite{pregated, expertflow, promoe} based on anticipated future needs, but suffer from critical limitations:
\textbf{First}, they optimize individual bottlenecks in isolation rather than addressing the coupled nature of prediction accuracy, memory bandwidth, and cache utilization.
\textbf{Second}, existing predictive systems rely on heuristics---stride-based patterns~\cite{promoe} or single-layer lookahead~\cite{expertflow}---that fail to capture complex cross-layer routing dependencies.
\textbf{Third}, none provide theoretical justification for why prediction should work or quantify the bounds on achievable accuracy.

\noindent\textbf{Problem Scope.}
Prophet targets latency-critical single-request inference (batch size 1), where each token's time-to-first-expert directly impacts user-perceived latency. In this regime, prediction accuracy is paramount: a single misprediction incurs the full 13$\times$--15$\times$ I/O penalty that dominates token generation time. Prophet additionally demonstrates throughput benefits for batched workloads in our evaluation, where expert reuse patterns provide complementary memory savings.

To address these challenges, this paper introduces \design, a neural expert prefetching system that transforms MoE inference from reactive execution to predictive optimization. Prophet's core innovation is a lightweight transformer-based predictor (8.4M parameters, <2\% overhead) that captures routing dependencies across multiple MoE layers, achieving 81--87\% prediction accuracy. High-accuracy predictions enable proactive expert loading that hides transfer latency behind computation, eliminating the I/O bottleneck rather than merely managing it. Prophet operates as a plug-and-play solution requiring no modifications to pre-trained MoE models, avoiding the prohibitively expensive process of retraining trillion-parameter models while enabling deployment across diverse hardware configurations.

\design makes the following {\bf contributions} to advance the state-of-the-art in MoE inference optimization:

\noindent$\bullet$ \textbf{Information-theoretic framework:} The first principled analysis quantifying expert routing predictability through mutual information (0.62 bits, 54\% uncertainty reduction) and power-law distributions ($\alpha = 1.2$--$2.2$), establishing theoretical bounds that guide predictor design choices.

\noindent$\bullet$ \textbf{Cross-layer neural prediction:} A learned neural predictor that captures cross-layer routing dependencies, achieving 81--87\% prediction accuracy through dense transformer architecture with layer encoding---advancing beyond heuristic approaches used in prior work.

\noindent$\bullet$ \textbf{Prediction-guided caching:} Two-tier hierarchical caching that maps prediction confidence to resource allocation, achieving 99.4\% effective hit rates by prioritizing high-confidence predictions for proactive loading.

\noindent$\bullet$ \textbf{Cross-architecture and cross-domain evaluation:} Comprehensive evaluation across five MoE architectures (Switch Transformer, Mixtral-8x7B, Qwen1.5-MoE, Qwen3-30B, DeepSeek-MoE-16B, GPT-OSS-20B) and six datasets, demonstrating $1.5\times$--$12.7\times$ performance speedups and robust cross-domain generalization where patterns learned from one dataset transfer effectively to others.

The remainder of this paper provides background on MoE architectures (\Cref{sec:background}), establishes theoretical foundations through information-theoretic analysis (\Cref{sec:Motivation}), details Prophet's system design and algorithms (\Cref{sec:design}), presents implementation and comprehensive experimental evaluation (\Cref{sec:implementation}), discusses related work in MoE optimization (\Cref{sec:related}), and concludes with implications for future MoE deployment (\Cref{sec:conclusion}).

%%%%%

%%%%%%%%%% %%%%%%%%%% %%%%%%%%%% %%%%%%%%%% 
\section{Background and Motivation}
\label{sec:background}

% Text matter begin
%%%%%%%%%% %%%%%%%%%% %%%%%%%%%% %%%%%%%%%% 

\subsection{MoE Fundamentals and Current Bottlenecks}
LLMs, are built from layers of transformer  (\Cref{fig:moe_architecture}(a)), each consisting of an attention mechanism (with linear projections), normalization layers, and a position-wise feedforward network, all connected by residual pathways. Classically, the feedforward network of a transformer block, in most LLM architectures, is a dense MLP block (\Cref{fig:moe_architecture}(b)). In contrast, MoE models modify this design (depicted in \Cref{fig:moe_architecture}(c),(d)) by replacing the monolithic feedforward MLP with a large set of expert networks and a learned router that activates only a sparse subset of them for each token, thereby increasing model capacity without a proportional increase in computational cost. 
Consequently, for an input token $\mathbf{x}^{(l)}$ at layer $l$, the router computes the expert selection probabilities through $G^{(l)}(\mathbf{x}^{(l)}) = \text{softmax}(\mathbf{W}_g^{(l)} \cdot \mathbf{x}^{(l)} + \mathbf{b}_g^{(l)})$, where $\mathbf{W}_g^{(l)}$ and $\mathbf{b}_g^{(l)}$ are learnable parameters. The routing network essentially learns which experts are most relevant for each input token based on its learned representations. The routing mechanism selects the top-$k$ experts with highest probabilities, forming $\mathcal{S}_{\text{selected}} = \text{top-k}(G^{(l)}(\mathbf{x}^{(l)}))$. The final output combines the outputs of selected experts weighted by their gating scores: $\mathbf{y}^{(l)} = \sum_{i \in \mathcal{S}_{\text{selected}}} G^{(l)}_i(\mathbf{x}^{(l)}) \cdot E_i^{(l)}(\mathbf{x}^{(l)})$, where each expert network $E_i^{(l)}$ processes tokens within its specialized domain, and the weighted combination dynamically blends the outputs of experts based on their relevance.

Modern MoE models exhibit significant architectural diversity in their routing strategies. Sparse routing approaches, like Switch Transformer~\cite{fedus2022switch}, employ top-1 (\Cref{fig:moe_architecture}(c)) expert selection, activating only 1 out of 128 experts per token. 
Dense routing architectures like Qwen1.5-MoE-A2.7B~\cite{bai2023qwen} employ 64 experts per MoE layer, comprising 4 shared always-active experts along with 4 out of 60 routing experts selected by the top-k router (8 experts per token), whereas Qwen3-235B-A22B~\cite{yang2025qwen3} scales this approach up to 128 experts with 8 activated per token (\Cref{fig:moe_architecture}(d));
architectural details are provided in \Cref{appedndix:MoEModels}.

\begin{figure}[h!]
\centering
\includegraphics[width=\linewidth]{figs/LLM-MoE-Arch.pdf}
\caption{LLM architecture comparison (a) Generic LLMs represented as layers of transformer blocks (b) inside view of a traditional dense LLM layer, (c) inside view of a MoE layer with top-1 routing (e.g. SwitchTransformer), and (d) MoE layer with top-k routing and shared experts where Expert-4 is always activated (e.g. Qwen1.5-MoE). The routing mechanism determines expert selection strategies and creates fundamentally different optimization challenges.}
\label{fig:moe_architecture}
%\vspace{-10pt}
\end{figure}

Despite computational efficiency advantages, MoE models suffer from severe memory and bandwidth bottlenecks driven by high I/O-to-compute ratios. For example, transferring a 6.3\,MB Switch Transformer~\cite{fedus2022switch} expert over PCIe~4.0 takes $197\,\mu$s versus $15\,\mu$s for computation ($13\times$ slower), while loading a 386\,MB Qwen1.5-MoE-A2.7B~\cite{bai2023qwen} expert requires 12\,ms versus 0.8\,ms ($15\times$ slower). Our experiments on both sparse and dense routing models using classical LLM datasets~\cite{googleNatural, wiki, IMDB, CNN}, suggest that I/O dominates 78--96\% of execution time (creating an execution bottleneck, as shown in \Cref{fig:io_bottleneck}) across different batch sizes.

State-of-the-art solutions address these bottlenecks by deploying multiple strategies like caching~\cite{promoe, moeinfinity, zhong2025hybrimoe}, pre-fetching~\cite{bambhaniya2024moeeras, pregated,doucet2025harmoeny}, and load-balancing~\cite{prophet2, ma2025moegpsguidlinespredictionstrategy, huang2023towards, wang2024auxiliary, expertflow}, but lack end-to-end optimization. For example, reactive caching uses on-demand expert loading with LRU replacement~\cite{hobbit, bambhaniya2024moeeras}. In comparison, ExpertFlow~\cite{expertflow} employs predictive loading for $2$--$10\times$ speedups, but its effectiveness degrades beyond 32 experts. On the other hand, ProMoE~\cite{promoe} achieves $2.2\times$ prefill and $2.07\times$ gains via proactive caching, targeting single-token optimization. FateMoE~\cite{fate} delivers $4.5\times$ speedups on edge GPUs but is tailored to operate in constrained settings. Pre-gated MoE~\cite{pregated} attains a near-optimal performance within $1.23\times$ of ideal but requires architectural changes.

\subsection{The Case for Intelligent Expert Prefetching}
The severe I/O bottlenecks motivate proactive expert loading strategies that can hide memory transfer latency behind computation~\cite{pregated}. Rather than waiting for routing decisions to materialize, intelligent prefetching can preload experts based on predicted future needs, potentially eliminating the $13\times$--$15\times$ I/O penalties that currently dominate execution time. However, effective prefetching requires understanding whether expert routing patterns contain an \textbf{exploitable structure}. If routing decisions were truly random, prediction-based prefetching would not provide any benefit over simple caching strategies. 
Existing studies on strategies such as predictive pre-fetching~\cite{bambhaniya2024moeeras, pregated,doucet2025harmoeny} or expert caching~\cite{promoe, moeinfinity, zhong2025hybrimoe} do not offer a theoretical or empirical justification for their chosen methods, lacking a clear understanding of why \emph{predictive} pre-fetching is necessary or how to optimally design such predictors; this absence of foundational insights motivates our comprehensive analysis of expert routing patterns in real-world MoE deployments across diverse datasets.


\section{Information-Theoretic Analysis}
\label{sec:Motivation}
To rigorously establish the feasibility of intelligent expert pre-fetching, we conducted a comprehensive analysis of expert routing patterns across three benchmark datasets: Google Natural Questions (factual QA)~\cite{googleNatural}, IMDb (sentiment analysis)~\cite{IMDB}, and CNN (abstractive summarization)~\cite{CNN}, which collectively span a wide diversity of topics and tasks, thereby providing a more realistic evaluation context. With the context window constrained to 256 tokens, we exhaustively logged, for each input sequence, the expert-routing decision at every decoded token, thereby yielding 37,200 execution traces and approximately 3.34 million expert selections across mixture-of-experts models in the Switch Transformer and Qwen-MoE families. Our findings uncover fundamental structural patterns that existing methods fail to exploit jointly, offering strong motivation for neural predictor approaches over simple frequency-based baselines.


\begin{figure}[h!]
  \centering
  \subfloat[Expert popularity]{%
    \includegraphics[width=0.5\linewidth]{figs/all_layers_expert_popularity.pdf}%
    \label{fig:power_law}
  }%\hfill
  \subfloat[Expert de-duplication]{%
    \includegraphics[width=0.5\linewidth]{figs/batch_scaling_theory_vs_empirical_logx_bold.pdf}%
    \label{fig:batch_scaling_validation}
  }
  \vspace{-10pt}
  \caption{
  \textbf{(a)} Expert popularity follows a heavy-tailed distribution across layers: the top 20\% of experts handle up to 75\% of routing decisions; this skew implies persistent load imbalance and cache/reuse opportunities for popular experts.
  \textbf{(b)} Batch scaling validation for expert de-duplication: empirical measurements 
  supports the scaling law that de-duplication efficiency improves with batch size.
}
%\vspace{-25pt}
\end{figure}
  

\subsection{Heavy-Tailed Expert Popularity Distribution}
To assess whether experts exhibit “hot” and “cold” behavior across layers, we conducted a power-law analysis on ranked expert usage~\cite{ghorbani2022scaling}. Specifically, we ranked experts by selection frequency and applied maximum-likelihood estimation to fit a power-law model. We found that expert selection follows a robust power-law distribution with exponent \(\alpha = 1.293 \pm 0.05\) across layers and workloads as depicted in ~\Cref{fig:power_law}. This heavy-tailed behavior results in extreme concentration: the top 20\% of experts account for up to 75\% of all routing decisions. {This is supported by the sweep study of unique number of experts activated for a token through all the layers of the Qwen 1.5 MoE for different batch sizes as depicted in \Cref{fig:batch_scaling_validation}. Here we observe that the unique number of experts activated (plotted along the y-axis) grows sub-linearly with the increase in batch size (x-axis is in log scale). 
The power-law structure approximates sublinear scaling of the expected number of unique experts accessed as a function of batch size}
motivating a batch-aware deduplication strategy that selects only unique experts and thereby reduces bandwidth requirements. We established this relation by fitting the empirical curve of unique experts versus batch size to a power-law model ($\mathbb{E}[U(B)] \sim C \cdot B^{0.932}$), and confirm its accuracy in \Cref{fig:batch_scaling_validation} showing
significant deduplication opportunities, especially at higher batch sizes.

These power-law properties enable universal, layer independent predictive prefetching strategies that achieve high hit rates by targeting consistently popular experts, irrespective of workload variation. Crucially, however, simple frequency-based methods fall short: due to temporal fluctuations and layer-specific specialization in expert popularity, naive ranking yields only ~12.3\% prediction accuracy, underscoring the need for more sophisticated neural modeling.


\subsection{Temporal Correlations in Expert Routing}
From the power-law analysis, we identify the presence of both hot and cold experts within each layer. To further investigate the structure of expert utilization, we examine whether there exists a temporal correlation in expert selection across layers. Specifically, we quantify the dependency between future expert choices and their historical routing context by computing the mutual information between the expert selected at a future layer $E_{\ell+h}$ and the sequence of experts selected in the preceding $c$ layers $E_{\ell-c+1:\ell}$. Formally, we estimate
$I(E_{\ell+h}; E_{\ell-c+1:\ell}),$
using a discrete mutual information estimator:
  $I(X;Y) = \sum_{x,y} p(x,y) \log_2 \frac{p(x,y)}{p(x)p(y)}$
  where $p(x,y)$ represents the joint probability distribution of context-target pairs 
  {and $p(x)$, $p(y)$ are the marginal distributions.
We observe that expert selections are substantially more predictable than random chance.
Simply put, knowing the recent routing history makes future expert selections 54\% more predictable than randomly guessing, providing a strong foundation for accurate neural prediction models. Multi-step prediction using 16-token context achieves 38\% higher information gain than single-step approaches, demonstrating that neural sequence models can capture multi-step dependencies that simple heuristics miss, enabling prediction accuracies approaching information-theoretic limits.}

The temporal structure exhibits complex dependencies that vary across layers and input types. The early layers show stronger syntactic correlations with recent token history, focusing on surface-level patterns like part-of-speech and basic syntax. Deeper layers exhibit semantic dependencies that span longer contexts, capturing higher-level meaning and discourse structure~\cite{liu2024fantastic, rajbhandari2022deepspeed, zhang2024harder}. This layer-dependent specialization suggests that effective prediction models should account for the hierarchical nature of transformer processing, motivating prediction architectures that can capture dependencies across multiple layers simultaneously to leverage both syntactic and semantic routing patterns.



\subsection{Design Requirements and Path Forward}
Current approaches face fundamental limitations preventing comprehensive MoE optimization, encountering critical trade-offs between prediction accuracy and computational overhead while optimizing individual bottlenecks in isolation rather than addressing the coupled nature of memory bandwidth, computational efficiency, and prediction accuracy. The convergence of power-law popularity, mutual information and temporal correlations, 
%and spatial clustering 
creates opportunities for comprehensive optimization that existing solutions cannot provide, demanding prediction mechanisms that exploit multi-step temporal correlations through neural sequence modeling while operating within information-theoretic bounds, cross-architecture generalization through adaptive models that handle different routing strategies, and batch-level optimization that leverages sublinear scaling properties with theoretical guarantees.

Our analysis transforms expert prefetching from engineering heuristic to theoretically-grounded optimization problem with quantifiable benefits. 
%The convergence of 
These insights motivates our \design framework: \emph{a neural expert prefetching system that unifies learned cross-layer prediction with intelligent batch optimization, providing the first comprehensive solution to the expert loading challenge in modern MoE inference.} By simultaneously exploiting spatio-temporal structure through neural sequence modeling,
%, spatial relationships through cluster-aware optimization
and batch-level patterns through empreically observed deduplication, \design addresses the fundamental limitations that prevent existing approaches from achieving optimal MoE inference performance.
%%%%%%%%%% %%%%%%%%%% %%%%%%%%%% %%%%%%%%%% 
%%%%%%%%%% %%%%%%%%%% %%%%%%%%%% %%%%%%%%%% 
\section{Design of Prophet}
\label{sec:design}

%%%%%%%%%%%%%%%%%%%%%%%%%%%%%%%%%%%%%%%%%%%%%%%%%%%%%%
Prophet integrates two key components grounded in our information-theoretic analysis (\Cref{sec:Motivation}): a \textbf{neural expert predictor} that forecasts expert needs multiple layers ahead by exploiting cross-layer routing dependencies, and a \textbf{prediction-guided hierarchical cache} that maps prediction confidence to cache tier placement. Prophet operates transparently at inference time without requiring modifications to pre-trained MoE models.


\subsection{Neural Expert Prediction}
\label{subsec:neural-predictor}
The temporal correlations and layer-wise specialization patterns identified in our analysis motivate a neural sequence modeling approach. We employ a lightweight dense transformer to capture dependencies between routing decisions across multiple transformer layers, learning that tokens routed to certain experts in early layers are likely to require specific experts in deeper layers.

\noindent\textbf{Comparison with Prior Predictive Approaches:}
Existing systems employ cross-layer information through various mechanisms: FATE~\cite{fate} uses cross-layer gates for predictive scheduling, MoE-Infinity~\cite{moeinfinity} traces activations across layers, and AdapMoE~\cite{adapmoe} performs single-layer lookahead prefetching. However, these approaches are fundamentally \textit{heuristic}: they exploit cross-layer information without quantifying its predictive value or optimizing the prediction mechanism itself. \Cref{tab:predictor_comparison} summarizes key differences.

\design differs in three key aspects: (1)~\textbf{Information-theoretic grounding:} Our analysis in \Cref{sec:Motivation} establishes that 2.59 bits (63\%) of predictive information comes from the 3-layer expert history---providing principled justification for context window selection rather than ad-hoc tuning. (2)~\textbf{Learned sequence modeling:} Rather than fixed heuristics, we train a dense transformer to learn complex routing dependencies, achieving 86\% prediction accuracy compared to 12.3\% for frequency-based baselines. (3)~\textbf{Confidence-calibrated prefetching:} Prediction confidence scores enable tiered cache placement, maximizing GPU memory utilization for high-certainty predictions.

\begin{table}[t]
\centering
\small
\begin{tabular}{lccc}
\toprule
\textbf{System} & \textbf{Prediction Method} & \textbf{Context} & \textbf{Top-1 Acc.} \\
\midrule
ProMoE~\cite{promoe} & LRU-based & Current layer & $\sim$65\% \\
PreGated~\cite{pregated} & Router lookahead & 1 layer & $\sim$70\% \\
FATE~\cite{fate} & Cross-layer gates & Heuristic & $\sim$72\% \\
AdapMoE~\cite{adapmoe} & Single lookahead & 1 layer & $\sim$74\% \\
\midrule
\textbf{\design} & Neural transformer & 3 layers & \textbf{86\%} \\
\bottomrule
\end{tabular}
\vspace{-5pt}
\caption{Comparison with prior predictive approaches. \design's learned multi-layer context achieves 12--21\% higher prediction accuracy than heuristic methods. Baseline accuracies measured under identical conditions using our reimplementation on routing traces.}
\label{tab:predictor_comparison}
\vspace{-10pt}
\end{table}

%%%%%%%%%%%%
\subsubsection{Formalization and Architecture Overview}

We formalize expert prediction as a sequence modeling task that predicts future expert selections for individual token positions based on historical routing  (the collected traces as described in \Cref{sec:Motivation}) and current hidden states. For a specific token at sequence position $t$ currently being processed at layer $\ell$, our goal is to predict which expert will be selected for this same token at a future layer $\ell+h$, where $h$ represents the prediction horizon.


\noindent\textbf{Input Representation:} The predictor input $\mathcal{H}_t$ for token position $t$ comprises three components:
$\mathcal{H}_t = \{E_t^{(\ell-c+1:\ell)}, L^{(\ell-c+1:\ell)},$$H_t^{(\ell)}\} $
% \begin{align}
% \mathcal{H}_t &= \{E_t^{(\ell-c+1:\ell)}, L^{(\ell-c+1:\ell)}, H_t^{(\ell)}\} \label{eq:predictor_input}
% \end{align}
where $E_t^{(\ell-c+1:\ell)} = [e_t^{(\ell-c+1)}, e_t^{(\ell-c+2)}, \ldots, e_t^{(\ell)}]$ represents the expert routing history for token $t$ over the past $c$ layers, $L^{(\ell-c+1:\ell)} = [\ell-c+1, \ell-c+2, \ldots, \ell]$ encodes the layer positions within this context window, and $H_t^{(\ell)} \in \mathbb{R}^d$ denotes the current hidden state for token $t$ at layer $\ell$ that will be processed by future routers. While our information-theoretic analysis (\Cref{sec:Motivation}) identifies attention patterns as a potential information source (0.75 bits), hidden states $H_t^{(\ell)}$ already encode attention-relevant information through the transformer's residual connections---the hidden state at layer $\ell$ integrates all preceding attention computations, making explicit attention features redundant and simplifying the predictor architecture.
%\noindent\textbf{Prediction Target:} 
The predictor output is the predicted expert selection:
%\begin{align}
% 
$\hat{e}_t^{(\ell+h)} = \text{Predictor}(\mathcal{H}_t; \theta)$ 
%\label{eq:predictor_output}
%\end{align}
where $\theta$ represents the learnable parameters of the neural predictor model, including embedding matrices, transformer weights, and prediction head parameters.
%\noindent\textbf{Batch Processing:} 
For a batch of $B$ tokens
%$\{t_1, t_2, \ldots, t_B\}$ 
at layer $\ell$, we generate predictions for all tokens in the same positions simultaneously, enabling batch-level optimization and deduplication of predicted expert requirements.

\noindent\textbf{Context Window Selection ($c$):} The context window $c$ determines how many previous layers' routing decisions to incorporate. Our empirical analysis reveals that $c = 3$ layers provides optimal prediction accuracy while maintaining computational efficiency. Longer contexts ($c > 5$) yield diminishing returns due to the exponential decay of temporal correlations identified in \Cref{sec:Motivation}, while shorter contexts ($c < 2$) lack sufficient history for effective pattern learning.

\noindent\textbf{Prediction Horizon Selection ($h$):} The prediction horizon $h$ must provide sufficient lead time for expert prefetching while maintaining prediction accuracy. The optimal horizon depends on three factors: (1) expert loading latency determined by expert size and memory bandwidth, (2) non-expert computation time available for hiding I/O latency, and (3) prediction accuracy degradation with increasing horizon. Critically, the predictor outputs predictions for horizons $h \in \{1, 2, 3\}$ simultaneously rather than a single fixed horizon. This multi-horizon capability enables the caching system to orchestrate prefetching across memory tiers: immediate predictions ($h=1$) target L1 cache for experts needed imminently, while longer horizons ($h=2,3$) enable proactive loading from L2 cache or host memory for experts further ahead in the execution pipeline. For typical MoE models (e.g., 386MB Qwen experts), $h = 3$ provides 15--20ms lead time, sufficient to hide expert loading behind computation while maintaining high prediction accuracy.

\noindent\textbf{Architecture Design:} Our predictor employs a dense transformer architecture specifically designed for cross-layer routing prediction. The architecture avoids using MoE models to predict MoE routing, eliminating recursive complexity while enabling efficient pattern learning. Expert identifiers are embedded into dense representations through learned embedding matrices $\mathbf{W}_E \in \mathbb{R}^{N_E \times d}$, where $N_E$ denotes the expert vocabulary size (128 for Switch Transformer, 64 for Qwen1.5 MoE, 256 for Qwen3 MoE) and $d = 320$ represents the embedding dimension chosen to balance representational capacity with computational efficiency.

\noindent\textbf{Layer Position Encoding:} To capture the hierarchical structure of transformer processing, we employ layer position encodings that enable the model to understand the significance of routing decisions at different depths. Each layer index $\ell$ in the context window is mapped to a learned embedding vector $\mathbf{W}_L[\ell] \in \mathbb{R}^d$, allowing the predictor to weight routing patterns differently based on their layer positions. Early layers typically exhibit syntactic routing patterns while deeper layers capture semantic dependencies, and our layer position encodings enable the model to automatically learn these layer-specific characteristics.

\begin{figure}[t]
\centering
\includegraphics[width=\linewidth]{figs/pipeline.drawio.pdf}
\caption{Neural expert predictor scheduling, illustrating routing history from layers $\ell-2$ to $\ell$ combined with current hidden states enables prediction of experts at layer $\ell+3$. 
}
\label{fig:predictor_architecture}
\end{figure}

\noindent\textbf{Toy Example:} Consider predicting expert selection for token $t$ at layer 8 while currently processing layer 6 (\Cref{fig:predictor_architecture}). The predictor input includes: expert IDs selected for token $t$ at layers 4, 5, and 6 (routing history $E_t^{(4:6)}$), layer position encodings $L^{(4:6)} = [4, 5, 6]$, and the hidden state $H_t^{(6)}$ that will be processed by the layer 8 router. The prediction $\hat{e}_t^{(8)}$ identifies which expert token $t$ will likely require at layer 8, enabling prefetching with a 3-layer horizon that provides sufficient lead time for expert loading.
This formalization enables systematic optimization of expert prefetching through neural sequence modeling while maintaining compatibility with existing pre-trained MoE models and providing theoretical guarantees on prediction performance through the information-theoretic bounds established in \Cref{sec:Motivation}.
%%%%%%%%%%%%

\subsubsection{Neural Predictor Architecture and Training}
Building on the problem formalization above, we now detail the neural predictor architecture and training methodology that realizes the expert prediction framework. Our predictor employs a dense transformer architecture specifically optimized for cross-layer routing prediction, avoiding the recursive complexity of using MoE models to predict MoE routing while enabling sophisticated pattern learning through multi-head self-attention mechanisms.

\noindent\textbf{Architecture Specification:} The predictor comprises 4 transformer layers with 10 attention heads each, utilizing model dimension $d_{\text{model}} = 320$ and feed-forward dimension $d_{ff} = 1280$. 
The architecture parameters were determined through extensive ablation studies balancing prediction accuracy with computational efficiency, ensuring the predictor overhead remains below 1\% of expert execution time.

\noindent\textbf{Input Processing and Embeddings:} Following the input formulation $\mathcal{H}_t = \{E_t^{(\ell-c+1:\ell)}, L^{(\ell-c+1:\ell)}, H_t^{(\ell)}\}$, the architecture processes each component through specialized embedding layers. Expert identifiers $E_t^{(\ell-c+1:\ell)}$ are mapped to dense representations via learned embedding matrix $\mathbf{W}_E \in \mathbb{R}^{N_E \times d_{\text{model}}}$, where each expert ID receives its own 320 dimensional representation. 
The layer position encodings 
are processed through a learned embedding matrix. This learned encoding enables the model to understand the hierarchical significance of routing decisions at different transformer depths, automatically learning that early layers capture syntactic patterns while deeper layers encode semantic dependencies. The hidden states $H_t^{(\ell)}$ undergo linear projection to match the model dimension before being combined with the embedded routing history and position encoding.

\noindent\textbf{Training Data Collection:} We collect expert routing traces from pre-trained MoE models processing six diverse datasets (\Cref{sec:Motivation}): Natural Questions, SQuAD, GSM8K, HumanEval, Alpaca, and WikiText. This diversity ensures the predictor learns model-intrinsic routing patterns rather than dataset-specific features. For each forward pass, we record expert selections and hidden states at every layer for each token position.

\begin{figure}[t]
\centering
\includegraphics[width=0.9\linewidth]{figs/switch_prediction_accuracy.pdf}
\vspace{-10pt}
\caption{Predictor accuracy vs batch size for Switch transformer against the state-of-the-art. We observe a rapid decline is the accuracy beyond the prediction horizon, h = 3.}
\label{fig:predictor_accuracy}
%\vspace{-20pt}
\end{figure}

\noindent\textbf{Composite Loss Function and Training Objectives:} The model training employs a composite loss function that simultaneously optimizes prediction accuracy, ranking consistency, and confidence calibration:
$\mathcal{L}_{\text{total}} = \lambda_1 \mathcal{L}_{\text{classification}} + \lambda_2 \mathcal{L}_{\text{ranking}} + \lambda_3 \mathcal{L}_{\text{confidence}}$. 
The primary classification loss uses cross-entropy to optimize expert prediction accuracy. The auxiliary ranking loss encourages correct relative ordering of expert selection probabilities, ensuring that the model not only predicts the most likely expert but also maintains meaningful probability rankings for alternative experts. The confidence calibration loss employs the Brier score to ensure that prediction confidence scores correlate with actual accuracy, enabling reliable dynamic prefetching decisions. 

\noindent\textbf{Optimization and Hyperparameter Selection:} Training employs the AdamW optimizer with $\beta_1 = 0.9$, $\beta_2 = 0.999$, weight decay of 0.01, and a learning rate schedule combining linear warmup over 500 steps followed by cosine annealing. The model achieves convergence within 15,000 training steps, typically requiring 3.5--4.5 hours on a single NVIDIA A100 GPU depending on the target MoE architecture and dataset size.
Through systematic hyperparameter exploration across context windows $c \in \{1, 2, 3, 4, 5\}$ and prediction horizons $h \in \{1, 2, \dots 10\}$, we empirically determine that $c = 3$ layers and $h = 3$ layers provide optimal performance (\Cref{fig:predictor_accuracy}). The 3-layer context window captures sufficient routing history to identify emerging patterns without excessive computational overhead or overfitting to noise. The 3-layer prediction horizon balances prediction accuracy with pre-fetching constraints: longer horizons provide more pre-fetching lead time but suffer from exponentially decreasing accuracy due to accumulated uncertainty, while shorter horizons offer higher accuracy but insufficient time for hiding memory transfer latency behind computation.
Most importantly, \Cref{fig:predictor_accuracy} shows that our predictor model is more accurate in predicting the experts for future layers compared to state-of-the-art models (ProMoE~\cite{promoe} and PreGated MoE~\cite{pregated}).

\noindent\textbf{Prediction Accuracy Definition:}
We define prediction accuracy as \textit{top-1 per-token-per-layer} accuracy: for each token $t$ at prediction horizon $h$, we check whether the predicted expert $\hat{e}_t^{(\ell+h)}$ matches the ground-truth expert selection $e_t^{(\ell+h)}$. This strict metric directly corresponds to prefetch utility---correct predictions enable cache hits, while incorrect predictions waste bandwidth. Across our evaluation, \design achieves 81--87\% top-1 accuracy across the five target architectures at horizon $h=3$, compared to 0.78--1.56\% random baselines---over $50\times$ improvement.

\noindent\textbf{Cross-Domain Generalization:}
A critical question is whether the predictor overfits to training domains or learns transferable routing patterns. We validate generalization by training on traces from a subset of domains and evaluating on held-out domains (\Cref{fig:cross_domain_transfer}). Results show prediction accuracy remains 67--98\% across all 36 train-test domain pairs, with only 3--8\% accuracy degradation compared to in-domain evaluation. This confirms that \design learns model-intrinsic routing structure---how transformer layers progressively transform representations---rather than dataset-specific lexical patterns. This robustness eliminates the need for per-workload predictor retraining.

\noindent\textbf{Predictor Overhead:} The 8.4M-parameter predictor introduces minimal overhead: $<$1\% latency (0.8ms prediction vs. 15--50ms MoE forward pass) and $<$2\% memory (34MB vs. 3.2--12.8GB MoE parameters). Training requires 3.5--4.5 hours on a single A100 and amortizes across all subsequent inference runs. Detailed overhead breakdown is provided in \Cref{subsec:predictor}.

% \subsection{Batch-Aware Optimization}
% \label{subsec:batch-optimization}
% \subsubsection{Expert Deduplication Strategy}
% The power-law distribution of expert popularity creates significant opportunities for batch-level optimization that prior single-token approaches cannot exploit. The concentration effects we identified, where the most popular experts handle a disproportionate fraction of routing decisions, suggest that multiple tokens in a batch will frequently request overlapping sets of experts (\Cref{fig:batch_scaling_validation}). Our deduplication algorithm, as explained in \Cref{alg:deduplication} in \Cref{app:memoryCaching}, and illustrated in \Cref{fig:batch-deduplication} exploits this pattern by identifying unique experts across all batch items and computing an optimal loading schedule that prioritizes high-frequency experts.
% The algorithm reduces the total number of expert loads from $O(B \cdot k)$ for naive per-item loading to $O(B^{\alpha} \cdot k)$ where $\alpha = 0.932 < 1$, providing sublinear scaling that becomes increasingly beneficial at larger batch sizes. This theoretical guarantee stems directly from the power-law scaling property, which we empirically validated, ensuring predictable performance improvements as batch sizes increase.

% \subsubsection{Hardware-Aware Transfer Scheduling}
% Beyond deduplication, we implement sophisticated transfer mechanisms that pipeline expert loading with computation to hide memory bandwidth bottlenecks. Rather than loading all experts before computation begins, we divide the transfers into chunks that can overlap with the ongoing computation.
% This pipelining reduces effective loading latency by hiding transfer costs behind computation, a technique particularly effective when combined with our predictive prefetching that provides sufficient lead time for memory operations. The scheduler incorporates hardware-specific bandwidth constraints to prevent memory bottlenecks through utility-based optimization that combines prediction confidence, expert frequency, and recency. Using prediction confidence scores, we initiate transfers for high-confidence predictions before routing decisions are finalized, with the confidence threshold dynamically adjusted based on available bandwidth and cache capacity to ensure the system adapts to varying hardware capabilities.

% \begin{figure}[t]
% \centering
% \includegraphics[width=0.7\linewidth]{figs/DeDuplication.pdf}
% \caption{Expert deduplication before transferring to the GPU memory reduces bandwidth requirements significantly.}
% \label{fig:batch-deduplication}
% \vspace{-10pt}
% \end{figure}

\subsection{Prediction-Guided Expert Caching}
\label{subsec:caching}

Building on established expert caching techniques~\cite{moeinfinity,promoe,prescope}, Prophet adapts hierarchical caching to leverage its neural predictor's calibrated confidence scores. The key insight is that prediction confidence directly indicates cache placement priority: high-confidence predictions ($>$0.8) warrant L1 (GPU) placement for immediate access, while medium-confidence (0.5--0.8) predictions target L2 (host memory) for speculative coverage. This prediction-driven placement contrasts with reactive policies (LRU/LFU) that cannot anticipate future expert needs.

The two-tier hierarchy maps naturally to GPU-host memory organization: L1 cache utilizes GPU memory (24--80GB HBM) with sub-microsecond access, while L2 cache leverages host memory (128--1024GB) with PCIe transfer latency ($\sim$200--500$\mu$s). Cache capacity allocation follows the hot-cold distribution: L1 holds 20--50 high-confidence experts---sufficient to cover the power-law ``head''---while L2 provides comprehensive coverage for the long tail.

\noindent\textbf{Distributed Deployment Benefits:} In pipeline-parallel and expert-parallel deployments where different GPUs handle different layers~\cite{expertaffinity, moetuner}, expert locality benefits compound. Experts assigned to a GPU's layers exhibit higher temporal reuse within that GPU, reducing cache thrashing from other layers' experts. The power-law concentration ensures that each GPU's ``hot'' experts remain stable across sequences, improving per-GPU cache efficiency without requiring cross-GPU coordination for cache management. Further details on replacement policies, promotion strategies, and confidence calibration are provided in \Cref{app:memoryCaching}.


\begin{figure}[t]
\centering
\includegraphics[width=0.9\linewidth]{figs/ProphetDesign.pdf.pdf}
\caption{Overall design of Prophet: For every layer, we run the expert prediction (\circled{1}) which goes for the expert selection logic to find the expert in the cache (\circled{2a}). If a hit, the expert is loaded to the LLM (\circled{3}); otherwise, upon a miss, the expert is looked up in the host memory (\circled{2b}) and unique experts are selected (\circled{3b}) and loaded (\circled{4b}) to the GPU and LLM (\circled{3}).}
\vspace{-10pt}
\label{fig:overallDesign}
%\vspace{-20pt}
\end{figure}

\subsection{System Integration}
\label{subsec:integration}

\noindent\textbf{Component Orchestration.}
\Cref{fig:overallDesign} illustrates Prophet's component pipeline. The neural predictor (\Cref{subsec:neural-predictor}) processes routing history from previous $c=3$ layers to generate expert predictions with calibrated confidence scores for horizons $h \in \{1,2,3\}$. These predictions drive the hierarchical cache system (\Cref{subsec:caching}), which maps confidence scores to cache tier placement: high-confidence predictions ($>0.8$) target L1 (GPU) for immediate access, medium-confidence ($0.5$--$0.8$) target L2 (host memory) for speculative coverage, while low-confidence predictions trigger on-demand loading.

\noindent\textbf{Performance Breakdown.}
Prophet achieves $\sim$98\% effective hit rates, decomposing into $\sim$85\% L1 hits (sub-$\mu$s access) and $\sim$13\% L2 hits ($\sim$200$\mu$s PCIe transfer). The remaining $\sim$2\% are cold misses (2--5ms). While both L1 and L2 hits avoid cold-miss penalties, L2 hits still incur transfer overhead, explaining why high hit rates do not translate linearly to speedup. The end-to-end speedup of $1.5\times$--$10.4\times$ further depends on baseline I/O dominance: GPT-OSS (92\% I/O) achieves 10.4$\times$ because eliminating I/O dominates execution, while Qwen1.5-MoE (36\% I/O) achieves only 1.5$\times$ because compute already constitutes the majority of baseline time.

\noindent\textbf{Scalability.}
Prophet scales across several dimensions. \textit{Model size:} The predictor's 8.4M parameters remain constant regardless of target MoE model size; only the expert embedding dimension changes (128--256 experts). \textit{Prediction horizon:} While $h=3$ provides optimal accuracy-latency tradeoff for current hardware, the architecture supports longer horizons with graceful accuracy degradation (\Cref{fig:predictor_accuracy}); as memory bandwidth improves, shorter horizons may suffice. \textit{Multi-GPU deployment:} In pipeline-parallel settings where GPUs handle specific layer ranges, per-GPU predictors can specialize on their assigned layers, reducing prediction complexity while exploiting layer-specific routing patterns.

% \noindent\textbf{Continuous Batching Integration.}
% Prophet's prediction mechanism is compatible with continuous batching frameworks (vLLM, TGI) that dynamically schedule requests for throughput optimization. In continuous batching, new requests join ongoing batches at iteration boundaries. Prophet accommodates this by: (1) maintaining per-request routing history that persists across batch iterations, (2) updating predictions as new tokens generate, and (3) aggregating cache priorities across concurrent requests to maximize shared expert reuse. The power-law expert concentration (\Cref{sec:Motivation}) ensures that popular experts remain in L1 cache across request boundaries, reducing cold-start penalties for newly-joined requests.

\noindent\textbf{Flexibility.}
Prophet adapts automatically to diverse MoE architectures. For sparse routing (top-1), precision-focused prefetching with aggressive L1 caching is most effective. For dense routing (top-4 to top-8), coverage-oriented strategies with multi-expert predictions provide better returns. The plug-and-play design requires only: (1) one-time predictor training from routing traces (3.5--4.5 hours), and (2) cache tier configuration based on available memory. No modifications to pre-trained MoE models are needed.
%%%%%%%%%% %%%%%%%%%% %%%%%%%%%% %%%%%%%%%% 
\begin{figure*}[t]
    \centering
    % Switch Transformer
    \subfloat[Switch Transformer]{%
        \includegraphics[width=0.32\linewidth]{figs/switch_memory_usage}%
        \label{fig:switch-memory-usage}
    }
    \hfill
    % Qwen1.5
    \subfloat[Qwen1.5 MoE]{%
        \includegraphics[width=0.32\linewidth]{figs/qwen15_memory_usage}%
        \label{fig:qwen15-memory-usage}
    }
    \hfill
    % Qwen3 MoE
    \subfloat[Qwen3 MoE]{%
     \includegraphics[width=0.32\linewidth]{figs/qwen3_memory_usage}%
        %\includegraphics[width=0.32\linewidth]{figs/qwen15_memory_usage}%
        \label{fig:qwen3-memory-usage}
    }
    \vspace{-10pt}
    \caption{Memory usage overhead comparison of baseline methods relative to Prophet across different MoE architectures. 
    Prophet's batch-aware deduplication reduces memory consumption: (a) on Switch Transformer, baselines use $1.5\times$–$2.5\times$ more memory; 
    (b) on Qwen1.5 MoE, competing methods consume $1.8\times$–$2.2\times$ more memory; 
    (c) on Qwen3 MoE, Prophet maintains efficiency at scale with up to $2.8\times$ memory savings.
    %, enabling deployment on resource-constrained hardware. 
    Here and after: \textcolor{red}{$\times$}: means the GPU hits OoM Error without optimization or the feature is unavailable. We ensure optimizing the memory footprint of the baselines where ever possible.
}
    \label{fig:combined-memory-usage}
\end{figure*}
\begin{figure*}[h]
    \centering
    % Switch Transformer
    \subfloat[Switch Transformer]{%
        \includegraphics[width=0.32\linewidth]{figs/switch_cache_hit_rate}%
        \label{fig:switch-cache-hit-rate}
    }
    \hfill
    % Qwen1.5
    \subfloat[Qwen1.5 MoE]{%
        \includegraphics[width=0.32\linewidth]{figs/qwen15_cache_hit_rate}%
        \label{fig:qwen15-cache-hit-rate}
    }
    \hfill
    % Qwen3 MoE
    \subfloat[Qwen3 MoE]{%
        \includegraphics[width=0.32\linewidth]{figs/qwen3_cache_hit_rate}%
        % \includegraphics[width=0.32\linewidth]{figs/qwen15_cache_hit_rate}%
        \label{fig:qwen3-cache-hit-rate}
    }
    \vspace{-10pt}
    \caption{Cache hit rates of different MoE expert caching strategies across MoE Models. 
    Prophet's neural prediction and hierarchical caching achieve superior hit rates: (a) on Switch Transformer, significantly outperforming simple policies (LRU, LFU) and advanced baselines (ProMoE, ExpertFlow); 
    (b) on Qwen1.5 MoE, sustaining 85–95\% hit rates across batch sizes; 
    (c) on Qwen3 MoE, delivering consistent 90–99\% hit rates, demonstrating scalability to larger architectures with hundreds of experts.}
    \label{fig:combined-cache-hit-rate}
\end{figure*}
 
\section{Implementation and Evaluation}
\label{sec:implementation}

% Text matter begin
%%%%%%%%%% %%%%%%%%%% %%%%%%%%%% %%%%%%%%%% 

\subsection{Implementation}
\label{subsec:implementation-details}

\noindent\textbf{Neural Predictor}
We implement the neural predictor as a 4-layer dense transformer with 8.4M parameters. The predictor processes the 3-layer routing history to predict expert selections for horizons $h \in \{1, 2, 3\}$ simultaneously, enabling the caching system to prefetch experts for multiple future layers---immediate predictions ($h=1$) target L1 cache, while longer horizons ($h=2,3$) enable proactive loading from L2 cache or host memory. Training uses routing traces from the same six datasets used in our information-theoretic analysis (\Cref{sec:Motivation}): Natural Questions, SQuAD, GSM8K, HumanEval, Alpaca, and WikiText. Details of the Neural Predictor are given in \Cref{subsec:neural-predictor} and \cref{app:predictor}.

\noindent\textbf{Baseline Systems}
We compare our proposal against six fundamentally different approaches to expert caching. \textbf{Traditional caching} (LRU, LFU) represents reactive strategies that load experts on-demand without prediction, relying solely on access frequency patterns that prove ineffective for dynamic expert routing.
\textbf{1. ProMoE}~\cite{promoe} employs stride-based prefetching, assuming regular access patterns by predicting that expert $E_i$ will be needed again after a fixed stride interval. This fails when routing exhibits complex temporal dependencies that require sequence modeling. \textbf{2. ExpertFlow}~\cite{expertflow} uses dynamic expert placement with token distribution analysis, but lacks cross-layer prediction capabilities and degrades beyond 32 experts due to capacity assumptions. \textbf{3. FATE}~\cite{fate} implements pipeline-aware scheduling optimized for edge deployment, using cross-layer gates but missing batch-level optimization entirely.
\textbf{4. PreGated-MoE}~\cite{pregated} represents architectural approaches where router $L$ predicts experts for layer $L+1$ during training, achieving near-optimal performance, but requiring costly model retraining. In contrast, \textbf{Prophet} operates as plug-and-play deployment, using cross-layer attention to capture dependencies that single-layer approaches miss, combined with batch deduplication that existing systems cannot exploit.

\noindent\textbf{Target Architectures}
We evaluate on five production MoE architectures representing the current state-of-the-art: Mixtral-8x7B, Qwen1.5-MoE-A2.7B, Qwen3-30B, DeepSeek-MoE-16B, and GPT-OSS-20B. These models exhibit diverse expert counts (64--256), routing strategies (top-4 to top-8), and architectural designs. We exclude Switch Transformer from our main evaluation as its top-1 routing is substantially easier to predict and already well-studied in prior work~\cite{promoe, pregated}; our target models with top-k routing present more challenging and comprehensive benchmarks that better represent modern MoE deployments.

\noindent\textbf{Implementation Details}
Our implementation comprises 15,200 \texttt{LoC} using PyTorch 2.0~\cite{pytorch} and HuggingFace Transformers~\cite{hftf}. We build on existing MoE infrastructures, instrumenting routing decisions through lightweight hooks. We faithfully reimplemented {\em all baselines}, ensuring fair comparison under identical hardware constraints. Our experiments used NVIDIA A100 (Mixtral), RTX 3090 (Qwen1.5), and H100 (Qwen3, DeepSeek, GPT-OSS) GPUs under iso-cache constraints: 20MB for Qwen1.5 MoE, 40MB for Mixtral, and 80MB for Qwen3/DeepSeek/GPT-OSS. We evaluated across batch sizes 1-64 (context length: 256) with 770+ configurations, reporting means over 150 runs with 95\% confidence intervals.


%%%%%%%%%% %%%%%%%%%% %%%%%%%%%% %%%%%%%%%%
\begin{figure*}[ht!]
    \centering
    % Switch Transformer
    \subfloat[Switch Transformer]{%
        \includegraphics[width=0.32\linewidth]{figs/switch_mean_latency}%
        \label{fig:switch-mean-latency}
    }
    \hfill
    % Qwen1.5 (first)
    \subfloat[Qwen1.5 MoE]{%
        \includegraphics[width=0.32\linewidth]{figs/qwen15_mean_latency}%
        \label{fig:qwen15-mean-latency}
    }
    \hfill
    % Qwen3 MoE
    \subfloat[Qwen3 MoE]{%
        %\includegraphics[width=0.32\linewidth]{figs/qwen3_mean_latency}%
        \includegraphics[width=0.32\linewidth]{figs/qwen3_mean_latency}%
        \label{fig:qwen3-mean-latency}
    }
    \vspace{-10pt}
    \caption{Mean latency overhead of baseline methods compared to Prophet across different MoE architectures. 
    Prophet consistently outperforms all baselines: (a) on Switch Transformer, achieving $4\times$ speedup over ProMoE and $2.0\times$--$2.2\times$ over FATE and ExpertFlow; 
    (b) on Qwen1.5 MoE, delivering $1.5\times$--$2.8\times$ speedup; 
    (c) on Qwen3 MoE, maintaining strong performance scaling with $2.1\times$--$4.5\times$ improvements, validating effectiveness across diverse expert configurations.}
    \label{fig:combined-mean-latency}
\end{figure*}
\begin{figure*}[h!]
    \centering
    % Switch Transformer
    \subfloat[Switch Transformer]{%
        \includegraphics[width=0.32\linewidth]{figs/switch_p99_latency}%
        \label{fig:switch-p99-latency}
    }
    \hfill
    % Qwen1.5
    \subfloat[Qwen1.5 MoE]{%
        \includegraphics[width=0.32\linewidth]{figs/qwen15_p99_latency}%
        \label{fig:qwen15-p99-latency}
    }
    \hfill
    % Qwen3 MoE
    \subfloat[Qwen3 MoE]{%
        \includegraphics[width=0.32\linewidth]{figs/qwen3_p99_latency}%
        %\includegraphics[width=0.32\linewidth]{figs/qwen15_p99_latency}
        \label{fig:qwen3-p99-latency}
    }
    \vspace{-10pt}
    \caption{P99 latency overhead of baseline methods compared to Prophet across different MoE architectures. 
    Prophet maintains consistently low tail latencies critical for production deployment: (a) on Switch Transformer, competing methods exhibit $1.5\times$–$2.8\times$ worse P99 performance; 
    (b) on Qwen1.5 MoE, Prophet achieves $1.6\times$–$2.5\times$ lower tail latencies; 
    (c) on Qwen3 MoE, Prophet delivers robust tail-latency guarantees with $1.8\times$–$3.2\times$ improvements 
    {(with additional memory optimizations)}.
    %, ensuring SLA compliance at scale.
    }
    %\vspace{-10pt}
    \label{fig:combined-p99-latency}
\end{figure*}
%%%%%%%%%% %%%%%%%%%% %%%%%%%%%% %%%%%%%%%%

%\vspace{-20pt}
\subsection{ Analysis of Results}
\label{subsec:results-analysis}

%\todo{Start with memory, then move to end2end latency -- what are we exactly measuring; for deduplication also gide the results; }
\noindent\textbf{Memory Efficiency and Cache Performance}
Prophet's memory efficiency advantage stems from exploiting power-law expert popularity through batch deduplication. For instance, as shown in ~\Cref{fig:combined-memory-usage}, at a batch size of 64, naive loading requires 256 expert transfers for Switch Transformer (4 experts/token × 64 tokens), but power-law concentration means only 32 unique experts are needed (87.6\% reduction). Baselines cannot exploit this structure, consuming $1.5\times$--$15\times$ more memory across different batch sizes and causing GPU memory bloating that forces smaller batch sizes or expert swapping.
Memory scaling follows our validated sublinear relationship because popular experts dominate routing decisions. Expert-23 appears in 45\% of routing decisions across all batch items, so loading it once serves multiple requests. ExpertFlow and ProMoE miss this optimization entirely, while traditional caching loads experts reactively without considering cross-batch reuse patterns.
Hierarchical cache performance (\Cref{fig:combined-cache-hit-rate}) demonstrates prediction quality through confidence calibration. Prophet's 99.4\% hit rate versus 65-80\% for baselines reflects three advantages: (1) \textbf{Prediction accuracy}: $\sim87$\% versus $\sim 60$\% enables better cache placement, (2) \textbf{Confidence estimation}: high-confidence predictions (72.3\% L1 hits) receive priority allocation, and (3) \textbf{Hierarchical coverage}: medium/low confidence predictions provide speculative coverage (27.1\% L2 hits) that traditional binary caching cannot achieve (Additional results in \Cref{sec:additional-results}).

\noindent\textbf{Memory Hierarchy Analysis}
Our analysis highlights Prophet's advantages using the metric of  \textit{AMATE} (Average Memory Access Time per Expert). We regard AMATE as a critical performance indicator because latency in Mixture-of-Experts (MoE) architectures is primarily constrained by the time required to fetch expert parameters. Operationally, AMATE quantifies the mean latency to locate the selected expert(s) and transfer their parameters from the host (CPU) memory to the device (GPU) memory, corresponding to the CPU-side stage depicted in the lower portion of \Cref{fig:predictor_architecture}.
%and \textit{TMAT} (Total Memory Access Time). 
%These capture complementary aspects of memory efficiency---per-expert vs. aggregate system behavior. 

\noindent{\underline{\textbf{AMATE:}} Prophet achieves $8\times$--$12\times$ better per-expert efficiency than baselines. At batch size of 1, improvements stem solely from cache effects, while at batch-32/64, deduplication accounts for $57$--$75\%$ of total gains, reflecting Prophet's ability to exploit expert popularity (top $20\%$ of experts serve $75\%$ of routes). Further details are given in \Cref{app:memory_hierarchy_analysis} and breakdown graphs are shown in \Cref{fig:amate-breakdown}.}


\noindent\textbf{Cross-Architecture Analysis}
Architecture-specific behaviors reveal why different MoE designs create distinct optimization challenges. \textbf{Switch Models} (sparse top-1 routing) shows the highest Prophet speedups ($4\times$ over ProMoE) because prediction errors have minimal impact (i.e., missing one expert prediction affects only one expert per token). The sparse routing pattern amplifies the benefits of prediction accuracy, making our 87\% accuracy highly effective.
\textbf{Qwen architectures} (dense top-k routing) exhibit different trends: smaller relative improvements ($2.8\times$ over baselines) but greater absolute memory savings due to increased expert overlap. Top-8 routing means that each token activates 8 experts, creating more deduplication opportunities when batch items share routing patterns. Prophet automatically adapts by predicting top-16 experts rather than top-3, ensuring adequate coverage for higher expert density.

\noindent\textbf{Baseline degradation patterns differ by architecture:} ExpertFlow performs reasonably on Switch Transformer but collapses on Qwen due to its 32-expert capacity assumption (Qwen has 64). ProMoE's stride-based prediction works marginally for Switch Transformer's more regular patterns but fails entirely on Qwen's complex top-k routing where expert selection depends on fine-grained probability distributions rather than deterministic top-1 choices.

\noindent\textbf{Ablation:} Component ablation reveals architecture dependent synergies: neural prediction provides consistent $1.8\times$--$2.4\times$ gains across all architectures, while batch de-duplication benefits $1.3\times$ gains on sparse Switch Transformer versus $2.1\times$ on dense Qwen. Hierarchical caching contributes uniformly ($1.2\times$--$1.7\times$) because confidence-based placement works regardless of routing strategy. The additive combination delivers $2\times$--$15\times$ improvements, with larger gains on architectures that provide more optimization opportunities.

\noindent\textbf{End-to-End Performance}
Prophet delivers $1.5\times$--$12.7\times$ speedups across all architectures (\Cref{fig:combined-mean-latency}), with performance scaling that reveals fundamental differences between approaches. {\em The key insight is that different baselines fail for different reasons:} ProMoE's stride-based prefetching assumes periodic patterns that do not exist in transformer routing, achieving $\sim 60$\% prediction accuracy versus our 87\%. ExpertFlow's single-layer optimization misses cross-layer dependencies, while FATE's edge focus limits aggressive prefetching when resources are available. Critically, baseline performance \textit{degrades} at higher batch sizes due to three factors: (1) \textbf{Cache thrashing}: traditional LRU/LFU policies cause increased misses as batch diversity grows, (2) \textbf{Bandwidth saturation}: reactive loading creates memory bottlenecks that worsen with batch size, and (3) \textbf{Missed deduplication}: baselines load identical experts multiple times per batch. Prophet's $1.8\times$--$4.2\times$ improvements over ProMoE stem from exploiting cross-layer dependencies: When layer 4 routes to expert-23, layer 7 predictably routes to expert-78, enabling proactive loading. P99 latency results (\Cref{fig:combined-p99-latency}) reveal the Prophet's critical advantage: $2\times$--$4.8\times$ tail latency improvements. Traditional systems exhibit variance due to unpredictable cache misses (a single missed expert causes $15\times$ loading delay). Prophet's confidence-based prefetching eliminates this variance by ensuring that high-probability experts are preloaded, maintaining consistent sub-millisecond access times.

\noindent\textbf{Cross-Domain Generalization}
A critical question for learned predictors is whether patterns generalize beyond training data. We evaluate Prophet's cross-domain robustness by training on a subset of datasets and testing on held-out domains. Our analysis across six datasets (Natural Questions, GSM8K, HumanEval, WikiText, Alpaca, SQuAD) reveals that routing patterns are predominantly \textit{model-intrinsic} rather than dataset-specific. The power-law exponent $\alpha$ varies by dataset (GSM8K: $\alpha=2.1$--$2.4$, more concentrated; HumanEval: $\alpha=1.2$--$1.9$, more diverse), but the underlying structure persists across domains. When training on two datasets and testing on a third, prediction accuracy drops only 3--8\% compared to in-domain evaluation. This robustness stems from Prophet learning layer-level routing progressions (syntactic→semantic→task-specific) that remain consistent regardless of input domain. The information-theoretic foundation (\Cref{sec:Motivation}) explains this: mutual information between cross-layer routing decisions reflects model architecture rather than input semantics, enabling generalization without per-dataset retraining.

\noindent\textbf{Predictor Overhead}
Prophet's neural predictor introduces minimal overhead: 8.4M parameters (0.1--0.2\% of typical MoE models), 0.15--0.18ms inference latency (<2\% of total inference time), and 3.5--4.2 hours training on A100 GPUs. Detailed overhead analysis is provided in \Cref{app:predictor}.

%%%%%
%%%%%%%%%% %%%%%%%%%% %%%%%%%%%% %%%%%%%%%% 

%\vspace{-10pt}
\section{Related Work} 
\label{sec:related} %Dee

% Text matter begin
%%%%%%%%%% %%%%%%%%%% %%%%%%%%%% %%%%%%%%%% 


%\textcolor{red}{Related works for: MoE, expert routing (prefetching, speculating), expert caching, expert load balancing, token to expert relationship, expert usage characterization}

The concept of employing ensembles of specialized neural networks, opportunistically used individually or in combination for task-specific computations, has garnered considerable attention in recent years. The technical details of this approach have been extensively discussed, from seminal work~\cite{jacobs1991adaptive, jordan1994hierarchical} to state-of-the-art research~\cite{jiang2024mixtral, deepseek2024v2, dai2024deepseekmoe}. We have arranged previous efforts in this direction into three categories: (\textbf{1}) expert prediction, \textbf{(2)} expert orchestration, and \textbf{(3)} expert usage characterization. 

\noindent\textbf{Expert Prediction:}
Prior work on expert prediction has evolved from reactive routing to proactive prediction strategies. Speculative MoE~\cite{li2025speculative} and DAOP~\cite{zhang2025daop} predict experts before routing decisions through speculative execution. Cross-layer approaches~\cite{chen2025crosslayer} and task-aware systems~\cite{wang2025emoe} leverage adjacent layer information, while MoE-GPS~\cite{ma2025moegpsguidlinespredictionstrategy} provides systematic predictor selection guidelines. Concurrent work DuoServe-MoE~\cite{duoserve} employs decode-stage lookahead predictors achieving up to 7.54$\times$ speedup on specific models. PreScope~\cite{prescope} introduces layer-group-aware predictors (LLaPor) achieving 94\% Top-4 accuracy by specializing architectures for input/middle/output layer groups, combined with global cross-layer scheduling. Prophet differs fundamentally: our information-theoretic analysis reveals that cross-layer mutual information is \textit{uniform across layer positions} (\Cref{sec:Motivation}), motivating a unified predictor that captures network-wide dependencies rather than fragmented layer-specific models. This principled design achieves 80--89\% Top-1 accuracy (a stricter metric than Top-4) while requiring a single model across all layers. Prophet further differs through (1) information-theoretic grounding that quantifies achievable prediction accuracy, (2) demonstrated cross-domain transfer (67--98\% accuracy across domain pairs), and (3) systematic evaluation across five diverse architectures.

\noindent\textbf{Expert Orchestration:}
Research efforts in the direction of orchestration of experts include addressing system-level challenges like housing the memory-intensive expert parameters in multi-tiered memory hierarchies, opportunistically caching the required experts in limited device memory along with minimizing communicational overhead~\cite{li2025comoe,suo2025coserve,kong2025serving,adapmoe}. Advanced caching systems include ProMoE~\cite{promoe}, MoE-Infinity~\cite{moeinfinity}, and HOBBIT~\cite{hobbit}. Learning-based caching approaches such as LRB~\cite{lrb} have demonstrated the potential of using machine learning to approximate optimal caching decisions in CDN contexts; Prophet extends this principle to MoE expert prediction with cross-layer dependencies. Distributed training optimizations are explored by FasterMoE~\cite{fastermoe}, Expert Choice Routing~\cite{expertchoice}, and MegaBlocks~\cite{megablocks}, focusing on load balancing and communication efficiency. An often overlooked but critical metric optimized by Prophet is the memory bandwidth and the count of unique experts that can be housed per inference. By combining high predictor accuracy with deduplication, Prophet loads more distinct experts into GPU memory without triggering out-of-memory events, while concomitantly reducing memory-bandwidth contention.

\noindent\textbf{CPU/GPU Hybrid Inference:}
Recent systems have explored heterogeneous execution to leverage both CPU and GPU resources for MoE inference. MoE-Lightning~\cite{cao2025moe} introduces CPU-GPU-I/O pipelining with hierarchical roofline modeling for memory-constrained GPUs, achieving up to 10$\times$ throughput gains. KTransformers~\cite{ktransformers} employs AMX-specialized CPU kernels with asynchronous scheduling for hybrid inference, demonstrating significant speedups on DeepSeek models. Prophet is \textit{orthogonal and complementary} to these approaches: while they optimize \textit{where} computation happens (CPU vs. GPU placement), Prophet optimizes \textit{when} experts are loaded through predictive prefetching. Prophet's predictions could guide CPU/GPU placement decisions, enabling combined deployment where predictions inform which experts to keep on GPU versus offload to CPU.

\noindent\textbf{Expert Usage Characterization:}
Prior work in this domain includes exploring the correlation among experts to complement optimization strategies~\cite{lo2025closerlookmixtureofexpertslarge, chen2024layerwise, zhang2024harder, oldfield2024multilinear}. The literature on specialization and redundancy spans DeepSeekMoE~\cite{dai2024deepseekmoe}, which enables fine-grained expert segmentation for behavioral profiling, and MoNE~\cite{zhang2024mone}, which addresses redundancy via a dual-metric assessment. Complementary work on semantic routing~\cite{chen2025probing} and correlation modeling~\cite{wang2024decorrelated} substantiates input-semantic routing and expert de-correlation techniques. In contrast to these primarily analytic approaches, Prophet couples such insights with system-level mechanisms, thereby translating them into actionable policies. Moreover, while correlation-driven studies are often model family-specific, Prophet leverages its context window to capture evolving temporal correlation characteristic of real-world deployments. Finally, an information-theoretic analysis of expert popularity reveals power-law scaling, conferring scalability across deployment regimes, and furnishing principled metrics for quantifying expert participation during inference.
%%%%%%%%%% %%%%%%%%%% %%%%%%%%%% %%%%%%%%%% 
\vspace{-10pt}
\section{Conclusions}
\label{sec:conclusion}

% Text matter begin
%%%%%%%%%% %%%%%%%%%% %%%%%%%%%% %%%%%%%%%% 
We addressed the fundamental memory bandwidth bottleneck constraining practical deployment of Mixture-of-Experts (MoE) models. Through information-theoretic analysis, we established that expert routing contains exploitable structure---power-law distributions and cross-layer dependencies providing 0.62 bits of mutual information---enabling accurate neural prediction of future expert requirements.

Building on these insights, \emph{Prophet} combines learned cross-layer prediction (80--89\% accuracy) with confidence-guided hierarchical caching, transforming MoE inference from reactive to predictive memory management. Our evaluation across five production architectures demonstrates $1.5\times$--$10.4\times$ TPOT speedups and $2.4\times$--$19.4\times$ tail latency (P99.9) improvements over on-demand loading, while enabling $4$--$8\times$ longer context lengths within the same memory budget. Critically, prediction accuracy remains 67--98\% across all cross-domain pairs, confirming that Prophet learns model-intrinsic routing patterns rather than dataset-specific features.

These results demonstrate that predictive memory management fundamentally shifts MoE execution from memory-bound to compute-bound, removing the dominant I/O barrier that previously limited deployment. The principles underlying Prophet---information-theoretic analysis guiding predictor design, cross-layer neural prediction, and confidence-aware resource allocation---hold promise for broader classes of sparse architectures where scalability and deployment efficiency are critical.
%%%%%%%%%% %%%%%%%%%% %%%%%%%%%% %%%%%%%%%% 
%%%%%%%%%% %%%%%%%%%% %%%%%%%%%% %%%%%%%%%%

% End Chapters

%%%%%%%%%% %%%%%%%%%% %%%%%%%%%% %%%%%%%%%%

%-------------------------------------------------------------------------------
\bibliographystyle{plain}
\bibliography{sample-base}

%%%%%%%%%% %%%%%%%%%% %%%%%%%%%% %%%%%%%%%%
%%%%%%%%%%   APPENDIX GOES HERE  %%%%%%%%%%
%\clearpage
%% If your work has an appendix, this is the place to put it.
\clearpage
\appendix
\pagenumbering{arabic}
\setcounter{page}{1}
%\section{Additional Results and Analysis}
\label{sec:additional-results}

This appendix provides comprehensive analysis of expert fetching patterns, memory bandwidth utilization, and deduplication efficiency across different MoE architectures. We present detailed performance characteristics that complement the main evaluation and provide deeper insights into Prophet's optimization strategies.

\subsection{Expert Fetching Analysis}
\label{subsec:expert-fetching-analysis}

Expert fetching represents a critical bottleneck in MoE inference, where the number of experts loaded from memory directly impacts both latency and memory bandwidth consumption. Our analysis reveals significant differences in expert fetching patterns across baseline approaches and demonstrates Prophet's effectiveness in minimizing expert access overhead.

\begin{figure*}[t]
    \centering
    % Switch Transformer
    \subfloat[Switch Transformer]{%
        \includegraphics[width=0.32\linewidth]{figs/switch_experts_fetched}%
        \label{fig:switch-experts-fetched}
    }
    \hfill
    % Qwen1.5
    \subfloat[Qwen1.5 MoE]{%
        \includegraphics[width=0.32\linewidth]{figs/qwen15_experts_fetched}%
        \label{fig:qwen15-experts-fetched}
    }
    \hfill
    % Qwen3 MoE
    \subfloat[Qwen3 MoE]{%
        \includegraphics[width=0.32\linewidth]{figs/qwen3_experts_fetched}%
        \label{fig:qwen3-experts-fetched}
    }
    \vspace{-10pt}
    \caption{Expert fetching overhead comparison across MoE architectures. 
    Prophet demonstrates substantial reductions in expert access: (a) Switch Transformer shows 2.1-2.8× lower expert fetching compared to baselines; 
    (b) Qwen1.5 MoE benefits from reduced fetching of larger experts (386MB vs. 6.3MB); 
    (c) Qwen3 MoE maintains efficient expert access patterns even with 256 experts, demonstrating scalability.
    Prophet's batch-level deduplication provides logarithmic scaling benefits that become increasingly pronounced at larger batch sizes.}
    \label{fig:combined-experts-fetched}
\end{figure*}

Figure~\ref{fig:combined-experts-fetched} demonstrates Prophet's superior expert fetching efficiency across all evaluated architectures. Key observations include:

\begin{itemize}
\item \textbf{Baseline Comparison}: ProMoE exhibits $2.1\times$ higher expert fetching overhead compared to Prophet due to proactive loading of potentially irrelevant experts. ExpertFlow shows $2.8\times$ overhead from migration-induced expert movements, while FATE demonstrates $2.3\times$ overhead from pipeline-aware scheduling complexities.

\item \textbf{Deduplication Impact}: Prophet's batch-level deduplication reduces expert fetching by 60-75\% compared to non-deduplicating approaches. This benefit scales logarithmically with batch size, as duplicate expert requests become more prevalent in larger batches.

\item \textbf{Architecture Scaling}: Prophet maintains consistent efficiency across different expert configurations: Switch (128 experts, 6.3MB each), Qwen1.5 (64 experts, 386MB each), and Qwen3 (256 experts), demonstrating broad applicability.
\end{itemize}

\subsection{Memory Bandwidth Analysis}
\label{subsec:memory-bandwidth-analysis}

Memory bandwidth represents a critical resource in MoE inference, with expert loading creating substantial data movement overhead. Our analysis examines how Prophet's architectural innovations impact memory subsystem utilization across different model configurations.

\begin{figure*}[t]
    \centering
    % Switch Transformer
    \subfloat[Switch Transformer]{%
        \includegraphics[width=0.32\linewidth]{figs/switch_memory_bandwidth}%
        \label{fig:switch-memory-bandwidth}
    }
    \hfill
    % Qwen1.5
    \subfloat[Qwen1.5 MoE]{%
        \includegraphics[width=0.32\linewidth]{figs/qwen15_memory_bandwidth}%
        \label{fig:qwen15-memory-bandwidth}
    }
    \hfill
    % Qwen3 MoE
    \subfloat[Qwen3 MoE]{%
        \includegraphics[width=0.32\linewidth]{figs/qwen3_memory_bandwidth}%
        \label{fig:qwen3-memory-bandwidth}
    }
    \vspace{-10pt}
    \caption{Memory bandwidth utilization comparison across MoE architectures. 
    Prophet achieves sub-linear bandwidth scaling: (a) Switch Transformer shows logarithmic bandwidth growth vs. linear scaling of baselines; 
    (b) Qwen1.5 MoE demonstrates 2.8× bandwidth reduction with larger experts (386MB); 
    (c) Qwen3 MoE maintains efficient bandwidth usage despite coordinating 256 experts.
    Prophet's deduplication and hierarchical caching reduce bandwidth requirements by 45-60\% across all architectures.}
    \label{fig:combined-memory-bandwidth}
\end{figure*}

Figure~\ref{fig:combined-memory-bandwidth} illustrates Prophet's memory bandwidth optimization across architectures:

\begin{itemize}
\item \textbf{Sub-Linear Scaling}: Prophet achieves sub-linear memory bandwidth growth with increasing batch size, contrasting with the linear scaling of baseline approaches without deduplication.

\item \textbf{Expert Size Impact}: Bandwidth benefits scale with expert size—Qwen1.5's 386MB experts amplify Prophet's 2.8× bandwidth reduction compared to Switch's 6.3MB experts, making optimization increasingly critical for larger models.

\item \textbf{Cache Hierarchy Effectiveness}: The hierarchical cache system reduces main memory pressure across all architectures, with L1/L2 caches serving frequently accessed experts and minimizing expensive DRAM transfers.
\end{itemize}

\subsection{Deduplication Efficiency Analysis}
\label{subsec:deduplication-analysis}

Batch-level deduplication represents a core innovation in Prophet, eliminating redundant expert loading within batches. Our analysis quantifies deduplication effectiveness across different model architectures and batch configurations.

\begin{figure*}[t]
    \centering
    % Switch Transformer
    \subfloat[Switch Transformer]{%
        \includegraphics[width=0.32\linewidth]{figs/switch_deduplication_efficiency}%
        \label{fig:switch-deduplication}
    }
    \hfill
    % Qwen1.5
    \subfloat[Qwen1.5 MoE]{%
        \includegraphics[width=0.32\linewidth]{figs/qwen15_deduplication_efficiency}%
        \label{fig:qwen15-deduplication}
    }
    \hfill
    % Qwen3 MoE
    \subfloat[Qwen3 MoE]{%
        \includegraphics[width=0.32\linewidth]{figs/qwen3_deduplication_efficiency}%
        \label{fig:qwen3-deduplication}
    }
    \vspace{-10pt}
    \caption{Deduplication efficiency comparison across MoE architectures. 
    Prophet achieves substantial redundancy elimination: (a) Switch Transformer shows 65-75\% deduplication at batch size 32; 
    (b) Qwen1.5 MoE demonstrates amplified benefits with larger experts (386MB vs. 6.3MB); 
    (c) Qwen3 MoE maintains 50-65\% efficiency despite 256 experts, proving scalability.
    Deduplication efficiency follows logarithmic scaling, with benefits increasing as expert overlap probability grows with batch size.}
    \label{fig:combined-deduplication}
\end{figure*}

Figure~\ref{fig:combined-deduplication} reveals deduplication effectiveness patterns across architectures:

\begin{itemize}
\item \textbf{Logarithmic Benefits}: Deduplication efficiency follows a logarithmic curve across all architectures, with larger batches providing proportionally greater benefits as expert overlap probability increases.

\item \textbf{Expert Size Amplification}: Larger experts amplify deduplication benefits—Qwen1.5's 386MB experts provide 60× greater bandwidth savings per deduplication event compared to Switch's 6.3MB experts.

\item \textbf{Scale Resilience}: Prophet maintains effective deduplication across different expert populations: Switch (128 experts), Qwen1.5 (64 experts), and Qwen3 (256 experts), demonstrating algorithmic scalability.
\end{itemize}

\subsection{Cross-Architecture Performance Insights}
\label{subsec:cross-architecture-insights}

Our comprehensive evaluation across Switch Transformer, Qwen1.5-MoE, and Qwen3-MoE reveals several critical insights about MoE inference optimization:

\subsubsection{Scaling Laws and Architecture Dependencies}

\begin{itemize}
\item \textbf{Expert Size Impact}: Memory bandwidth becomes increasingly critical as expert size grows. Qwen1.5's 386MB experts amplify Prophet's optimization benefits by $60\times$ compared to Switch's 6.3MB experts.

\item \textbf{Expert Population Scaling}: Prophet's hierarchical architecture scales effectively from 64 to 256 experts, maintaining performance benefits across different expert population sizes.

\item \textbf{Batch Size Optimization}: Deduplication benefits scale logarithmically with batch size across all architectures, suggesting optimal batch size selection strategies for different deployment scenarios.
\end{itemize}

\subsubsection{Hardware Resource Utilization}

\begin{itemize}
\item \textbf{Memory Hierarchy Effectiveness}: The hierarchical cache system provides consistent benefits across all evaluated architectures, with hit rates remaining above 85\% for L1 cache across models.

\item \textbf{Bandwidth Efficiency}: Prophet achieves $2.5\times$--$4.2\times$ memory bandwidth efficiency compared to baseline approaches, with benefits scaling with expert size.

\item \textbf{Prediction Accuracy Scaling}: Neural prediction maintains 82-88\% accuracy across different architectures, demonstrating the generalizability of the prediction approach.
\end{itemize}

\subsubsection{Deployment Implications}

\begin{itemize}
\item \textbf{Hardware Configuration Guidance}: Larger expert models (like Qwen1.5) benefit more from high-bandwidth memory subsystems, while smaller expert models (like Switch) can achieve good performance with optimized cache hierarchies.

\item \textbf{Batch Size Optimization}: Optimal batch sizes vary by architecture: Switch Transformer benefits from larger batches (32+) for deduplication, while Qwen models show benefits even at smaller batch sizes (8-16) due to larger expert granularity.

\item \textbf{Prediction Horizon Selection}: Different architectures benefit from different prediction horizons: Switch at h=2, Qwen1.5 and Qwen3 at h=3, suggesting architecture-specific tuning opportunities.
\end{itemize}

\subsection{Comparative Analysis with State-of-the-Art}
\label{subsec:comparative-analysis}

Our evaluation demonstrates Prophet's substantial improvements over existing approaches across all evaluated metrics:

\subsubsection{Latency Improvements}

\begin{itemize}
\item \textbf{Mean Latency}: Prophet achieves $1.5\times$--$3.2\times$ latency improvements compared to ProMoE, ExpertFlow, FATE, and PreGated-MoE across different architectures.

\item \textbf{P99 Latency}: Tail latency improvements of $2.3\times$--$6.2\times$ demonstrate Prophet's consistency and predictability advantages.

\item \textbf{Architecture Scaling}: Latency benefits increase with expert size, suggesting Prophet's effectiveness grows with model complexity.
\end{itemize}

\subsubsection{Memory Efficiency}

\begin{itemize}
\item \textbf{Memory Usage}: Prophet reduces memory consumption by $1.5\times$-$15\times$ compared to baseline approaches through effective deduplication and cache management.

\item \textbf{Expert Fetching}: Expert access overhead reduction of $2.1\times$--$4.2\times$ across architectures demonstrates Prophet's effectiveness in minimizing expensive memory operations.

\item \textbf{Bandwidth Utilization}: Sub-linear memory bandwidth scaling provides sustainable performance growth with increasing batch sizes.
\end{itemize}

\subsubsection{Cache Performance}

\begin{itemize}
\item \textbf{Hit Rates}: Prophet maintains 87-95\% cache hit rates across architectures, substantially outperforming traditional LRU/LFU approaches (45-50\%).

\item \textbf{Prediction Accuracy}: Neural prediction achieves 82-88\% accuracy across different architectures and prediction horizons, enabling effective prefetching strategies.

\item \textbf{Hierarchical Benefits}: The hierarchical cache system provides consistent performance benefits across all evaluated models and configurations.
\end{itemize}

\subsection{Key Insights and Discussion}
\label{subsec:insights-discussion}

Our comprehensive evaluation across three distinct MoE architectures validates several fundamental insights about expert routing optimization and reveals critical design principles for future MoE systems.

\subsubsection{Exploitable Structure in Expert Routing}

Our results validate that expert routing contains substantial exploitable structure across all evaluated architectures. The power-law popularity distribution $(\alpha = 1.293)$ remains consistent from Switch Transformer's 128 experts to Qwen3's 256 experts, suggesting universal optimization opportunities. This structure enables Prophet's batch deduplication to achieve 50-75\% redundancy elimination, translating directly to memory bandwidth savings that scale logarithmically with batch size.
The temporal correlations identified through mutual information analysis prove crucial for cross-layer prediction. Prophet's neural predictor achieves 82-88\% accuracy across architectures by exploiting these dependencies, demonstrating that routing patterns are far from random and contain learnable dependencies that span multiple transformer layers.

\subsubsection{Cross-Layer Dependencies and Architecture-Specific Adaptations}

Cross-layer dependencies matter significantly for effective prefetching. Our analysis reveals that different architectures benefit from different prediction horizons: Switch Transformer at h=2, while Qwen models achieve optimal performance at h=3. This suggests that expert size and routing complexity influence the optimal prediction distance.
Architecture-specific adaptations prove crucial: sparse routing (Switch) benefits from precision-focused prefetching with aggressive L1 caching, while dense routing (Qwen) requires coverage-oriented strategies with extensive deduplication. Prophet automatically adapts to these characteristics without requiring manual tuning, highlighting the importance of adaptive systems that adjust to routing patterns.

\subsubsection{Batch-Level Optimization Opportunities}

Batch-level optimization opportunities remain largely untapped by existing systems. Our deduplication algorithm delivers exponential memory savings at scale, reducing bandwidth requirements by 45-60\% across all architectures. The sub-linear scaling properties $(O(B^{0.932}))$ provide theoretical guarantees that become increasingly valuable as deployment scales increase.
Notably, many baseline models encounter GPU out-of-memory (OoM) errors at larger batch sizes. We adapted additional memory optimizations (context length reduction, staged memory loading) to enable fair comparison, but Prophet's superior memory efficiency allows larger batch sizes without these constraints.

\subsubsection{Deployment-Time vs. Training-Time Optimization}

Prophet's plug-and-play deployment contrasts favorably with architectural approaches like PreGated-MoE. Although PreGated-MoE delivers near-optimal theoretical performance, its reliance on costly retraining (requiring weeks of compute for trillion-parameter models) hampers practical adoption. Prophet demonstrates that deployment-time optimization can achieve competitive benefits ($1.5\times$--$3.2\times$ speedups) without infrastructure disruption or model modification.
This deployment flexibility becomes critical as MoE models continue scaling. The ability to optimize existing pre-trained models without retraining enables immediate deployment of optimization techniques across diverse model families and hardware configurations.

\subsubsection{Hardware-Software Co-Design Implications}

Our bandwidth analysis reveals that memory hierarchy design significantly impacts MoE performance. Qwen1.5's 386MB experts amplify Prophet's benefits by $60\times$ compared to Switch's 6.3MB experts, suggesting that future MoE architectures should consider expert size in relation to available memory bandwidth.
The consistent effectiveness of Prophet's hierarchical caching across different hardware configurations (A100, RTX 3090, H100) demonstrates the approach's hardware agnostic nature, but optimal cache tier sizes vary significantly with GPU memory capacity and PCIe bandwidth characteristics.

\subsection{Future Research Directions}
\label{subsec:future-directions}

Our analysis reveals several promising directions for future MoE optimization research:

\subsubsection{Advanced Prediction Mechanisms}

\begin{itemize}
\item \textbf{Multi-Modal Prediction}: Incorporating attention patterns, token-level context, and semantic relationships for improved expert prediction accuracy beyond current 82-88\% levels.

\item \textbf{Adaptive Prediction Horizons}: Dynamic horizon selection based on expert size, available bandwidth, and prediction confidence to optimize the accuracy-latency trade-off.

\item \textbf{Cross-Request Learning}: Leveraging patterns across multiple concurrent requests to improve prediction accuracy in multi-tenant deployment scenarios.
\end{itemize}

\subsubsection{System-Level Optimizations}

\begin{itemize}
\item \textbf{Distributed Expert Management}: Extension to multi-GPU and multi-node scenarios with coordinated caching and expert migration strategies.

\item \textbf{Quality-Performance Trade-offs}: Adaptive expert selection strategies that gracefully degrade quality under resource constraints while maintaining acceptable performance.

\item \textbf{Hardware Co-Design}: Custom memory hierarchies and interconnect designs optimized specifically for MoE inference patterns and expert routing characteristics.
\end{itemize}

\subsection{Conclusion}
\label{subsec:appendix-conclusion}

This comprehensive analysis demonstrates Prophet's effectiveness across diverse MoE architectures and provides deep insights into the mechanisms driving performance improvements. The consistent benefits across Switch Transformer, Qwen1.5-MoE, and Qwen3-MoE validate Prophet's architectural principles and highlight the critical importance of intelligent expert management in modern MoE inference systems.
Key findings include:
\begin{itemize}
\item \textbf{Universal Structure}: Power-law expert popularity $(\alpha = 1.293)$ and temporal correlations exist across all evaluated architectures, enabling consistent optimization opportunities
\item \textbf{Scalable Deduplication}: Batch-level deduplication provides logarithmic performance scaling with sub-linear memory growth $(O(B^{0.932}))$ across all architectures
\item \textbf{Cross-Architecture Prediction}: Neural prediction maintains high accuracy (82-88\%) across different model complexities and routing strategies
\item \textbf{Expert Size Amplification}: Larger expert parameters amplify optimization benefits, with Qwen1.5's 386MB experts providing 60× greater bandwidth savings per optimization compared to Switch's 6.3MB experts
\item \textbf{Hierarchical Cache Effectiveness}: Multi-tier caching provides consistent 85-95\% hit rates regardless of expert population size or architecture complexity
\item \textbf{Deployment Flexibility}: Plug-and-play optimization achieves $1.5\times$-$3.2\times$ speedups without requiring costly model retraining, enabling immediate deployment across diverse model families
\end{itemize}

These results establish Prophet as a robust, scalable solution for MoE inference optimization that adapts automatically to different architectural characteristics while providing theoretical guarantees on performance improvements. The insights from this analysis provide a foundation for future research in intelligent expert management and highlight the critical importance of system-level optimization in realizing the full potential of large-scale MoE models.

%\section{Memory Hierarchy Performance Analysis}
\label{app:memory_hierarchy_analysis}

This section provides a comprehensive analysis of Prophet's memory hierarchy optimizations through Average Memory Access Time per Expert (AMATE). 

\subsection{AMATE: Per-Expert Memory Access Efficiency}
\label{subsec:amate_analysis}

AMATE quantifies the average time required to access each expert from the memory hierarchy, accounting for both cache hit rates and expert deduplication efficiency. This metric captures per-expert efficiency by normalizing total memory access time by the number of unique experts accessed:

\begin{equation}
\text{AMATE} = \frac{\sum_{i} (\text{access\_time}_i \times \text{frequency}_i)}{\text{num\_unique\_experts}}
\end{equation}

Prophet's AMATE performance demonstrates systematic advantages across all evaluated architectures, with baselines exhibiting $8\times$--$12\times$ worse performance relative to Prophet's optimized memory hierarchy utilization.

\begin{figure*}[ht]
    \centering
    % Switch Transformer AMATE
    \subfloat[Switch Transformer AMATE]{%
        \includegraphics[width=0.32\linewidth]{figs/switch_amate_breakdown}%
        \label{fig:switch-amate}
    }
    \hfill
    % Qwen1.5 AMATE
    \subfloat[Qwen1.5 MoE AMATE]{%
        \includegraphics[width=0.32\linewidth]{figs/qwen15_amate_breakdown}%
        \label{fig:qwen15-amate}
    }
    \hfill
    % Qwen3 AMATE
    \subfloat[Qwen3 MoE AMATE]{%
        \includegraphics[width=0.32\linewidth]{figs/qwen3_amate_breakdown}%
        \label{fig:qwen3-amate}
    }
    \vspace{-10pt}
    \caption{AMATE breakdown across MoE architectures showing Prophet's per-expert memory access efficiency advantages. 
    Prophet achieves 8×--12× better AMATE performance through: (a) Switch Transformer benefits from precise prediction in sparse routing; 
    (b) Qwen1.5 MoE demonstrates consistent efficiency across dense routing patterns; 
    (c) Qwen3 MoE maintains scalability advantages with hundreds of experts. 
    The breakdown reveals growing deduplication benefits (dotted pattern) at higher batch sizes, reaching 57--75\% of total improvements at batch-64.
    \textcolor{red}{X}: OoM conditions where baselines exceed memory capacity.}
    \label{fig:amate-breakdown}
    \vspace{-10pt}
\end{figure*}

\subsubsection{AMATE Component Analysis}

The AMATE breakdown reveals, as depicted in \Cref{fig:amate-breakdown} two primary sources of Prophet's efficiency advantages:

\noindent\textbf{Caching Benefits (Diagonal Pattern):} Cache hit rate improvements contribute consistently across batch sizes, providing the foundational advantage through Prophet's prediction-driven prefetching. At batch-1, caching represents 100\% of AMATE benefits since deduplication opportunities don't exist. Prophet's neural prediction achieves 87\% expert prediction accuracy, enabling proactive loading to fast GPU memory (L1 cache) versus reactive loading from slower CPU memory (L2 cache). The $15\times$ latency difference between GPU and CPU memory access creates substantial per-expert efficiency gains.

\noindent\textbf{De-duplication Benefits (Dotted Pattern):} Expert deduplication becomes increasingly dominant at higher batch sizes, reflecting Prophet's exploitation of power-law expert popularity distributions. At batch-32 and batch-64, deduplication contributes 57--75\% of total AMATE improvements. This scaling occurs because Prophet identifies that popular experts (top 20\%) handle 75\% of routing decisions, enabling systematic reduction in unique experts loaded per batch. Baselines perform redundant expert loading due to poor cross-batch optimization, resulting in linear scaling of expert accesses rather than Prophet's sublinear $B^{0.932}$ scaling.

\noindent\textbf{Cross-Architecture Consistency:} AMATE benefits remain consistent across diverse MoE architectures, demonstrating Prophet's generalizability. Switch Transformer (sparse top-1 routing), Qwen1.5 MoE (dense top-4 routing), and Qwen3 MoE (dense top-8 routing) all exhibit similar $8\times$--$12\times$ AMATE advantages, indicating that Prophet's approach adapts effectively to different routing strategies and expert densities.

\noindent\textbf{Cache Penalties (Diagonal Pattern):} Cache-related penalties remain relatively stable across batch sizes, contributing 70--80\% of baseline penalties at batch-1 and 72--76\% at higher batch sizes. This stability reflects the fact that cache hit rates degrade gradually with increasing batch diversity, but the absolute impact scales linearly with expert access frequency. Prophet's predictive prefetching maintains high hit rates (85--95\%) compared to reactive baselines (10--65\%), directly translating to reduced aggregate memory access time.

% \noindent\textbf{Deduplication Penalties (Dotted Pattern):} Deduplication-related TMAT penalties exhibit systematic growth with batch size, increasing from 0\% at batch-1 to 24--28\% at batch-64. This growth pattern reflects the multiplicative nature of TMAT: each additional redundant expert access multiplies with longer access times. Prophet's power-law exploitation reduces unique expert requirements from linear $B^{0.98}$ scaling (baselines) to sublinear $B^{0.932}$ scaling, creating compounding benefits that become more significant at production-relevant batch sizes.


% \subsection{AMATE vs TMAT: Comparative Insights}
% \label{subsec:amate_tmat_comparison}

% The divergent behavior between AMATE and TMAT metrics provides complementary insights into Prophet's optimization approach:

% \noindent\textbf{Deduplication Impact Differences:} AMATE shows higher deduplication percentages (57--75\%) compared to TMAT (24--28\%) at batch-64. This difference occurs because AMATE normalizes by the number of unique experts accessed, amplifying the impact of expert deduplication, while TMAT captures absolute costs where cache penalties dominate due to per-access latency differences.

% \noindent\textbf{Scaling Behavior:} AMATE demonstrates more dramatic deduplication scaling (0\% to 75\%) compared to TMAT's gradual growth (0\% to 28\%). This reflects AMATE's sensitivity to expert count reduction versus TMAT's sensitivity to total time reduction, highlighting different aspects of Prophet's performance benefits.

% \noindent\textbf{Practical Implications:} TMAT provides more direct correlation with end-to-end latency improvements, making it the preferred metric for system-level performance analysis. AMATE offers insights into per-expert efficiency that guide algorithmic optimizations and resource allocation decisions.


%%%%%%%%%% %%%%%%%%%% %%%%%%%%%% %%%%%%%%%% 

\section{Neural Expert Predictor: Detailed Design}
\label{app:predictor}

This appendix provides comprehensive technical details of our neural expert predictor architecture, extending the overview presented in \Cref{sec:design} with implementation specifics, theoretical foundations, and design rationale.

\subsection{Theoretical Foundations}
\label{subsec:theoretical_foundations}

\subsubsection{Information-Theoretic Motivation}

The design of our neural predictor is grounded in the information-theoretic analysis presented in \Cref{sec:background}, which revealed that expert routing contains exploitable temporal structure. Specifically, pairwise mutual information between adjacent layers is $I(E^{(\ell)}; E^{(\ell+1)}) = 0.62 \pm 0.04$ bits. When expanded to incorporate a 3-layer context window with hidden states and attention patterns, our analysis reveals 4.11 bits (59\% of maximum entropy for 128 experts) are exploitable for prediction. MI values are computed via plug-in entropy estimators with Miller-Madow bias correction over 37,200 execution traces; bootstrap resampling (1000 iterations) provides 95\% confidence intervals.

This finding suggests that expert routing is neither completely random nor fully deterministic, but contains sufficient structure to enable meaningful prediction. The exponential decay of mutual information with increasing prediction horizon $h$ (following $I(h) \approx I_0 \cdot e^{-\lambda h}$ with decay constant $\lambda = 0.23$) provides theoretical bounds on prediction accuracy and guides our horizon selection strategy.

\subsubsection{Cross-Layer Routing Dependencies}

Our analysis of routing traces reveals hierarchical patterns in expert specialization that motivate the predictor architecture. Early transformer layers (layers 1-6) exhibit routing based primarily on syntactic features: part-of-speech tags, grammatical structure, and surface-level linguistic patterns. Middle layers (layers 7-15) show increasing semantic specialization, with experts focusing on entity types, conceptual categories, and basic relationships. Deep layers (layers 16+) demonstrate complex semantic routing based on discourse structure, pragmatic understanding, and task-specific reasoning.

These hierarchical patterns create predictable routing transitions that our neural predictor can exploit. For instance, in summarization tasks, tokens representing numerical quantities often route to syntactic experts in early layers (Expert-42 for numbers), then to semantic experts in middle layers (Expert-78 for statistical concepts), and finally to task-specific experts in deep layers (Expert-134 for performance metrics). The predictor learns to model these progression patterns through its cross-layer attention mechanism.

\subsection{Detailed Architecture Specification}
\label{subsec:architecture_details}

\subsubsection{Model Architecture Parameters}

Our neural predictor employs a carefully optimized dense transformer architecture with the following specifications:

\begin{itemize}
\item \textbf{Model Dimensions:} $d_{\text{model}} = 320$, chosen to balance representational capacity with computational efficiency. This dimension provides sufficient capacity to encode expert embeddings, layer position information, and hidden state projections while keeping the model lightweight relative to the main MoE inference.

\item \textbf{Transformer Layers:} 3 transformer layers, determined through ablation studies showing diminishing returns beyond 6 layers for the routing prediction task. Fewer layers ($< 3$) show reduced accuracy on complex routing patterns, while more layers (8-12) increase computational overhead without meaningful accuracy improvements.

\item \textbf{Attention Configuration:} 10 attention heads with $d_{\text{head}} = 32$, enabling diverse routing dependency patterns to be captured across different representational subspaces. The multi-head design allows specialization: some heads focus on short-term layer-to-layer transitions, others capture long-range dependencies, and specialized heads learn task-specific routing patterns.

\item \textbf{Feed-Forward Networks:} $d_{ff} = 1280$ ($4\times$ model dimension), following standard transformer scaling. The feed-forward layers enable non-linear combination of attention outputs and provide the capacity for complex routing pattern recognition.

\item \textbf{Parameter Count:} Total of 8.4M parameters distributed as follows: expert embeddings ($40.96K \times N_E$), position embeddings ($14.08K \times N_L$), transformer layers (7.2M), and prediction head (320K). This represents approximately 0.1--0.2\% of the parameters in typical billion-parameter MoE models.
\end{itemize}

\subsubsection{Input Representation and Embedding Strategy}

The predictor processes three distinct input modalities through specialized embedding strategies:

\paragraph{Expert Identity Embeddings} Expert identifiers are embedded through learned matrices $\mathbf{W}_E \in \mathbb{R}^{N_E \times d_{\text{model}}}$ where $N_E$ varies by architecture: 128 for Switch Transformer, 64 for Qwen1.5 MoE, and 256 for Qwen3 MoE. Each expert receives a unique 320-dimensional embedding initialized using Xavier uniform initialization scaled by $\sqrt{2/d_{\text{model}}}$.

The expert embedding strategy differs from traditional token embeddings in several key aspects: (1) Expert IDs represent categorical selections from routing decisions rather than sequential tokens, requiring different initialization strategies. (2) Expert embeddings must capture semantic similarity between functionally related experts, necessitating learning of expert clustering patterns. (3) The embedding space must be sufficiently expressive to distinguish between hundreds of experts while remaining computationally tractable.

\paragraph{Layer Position Encoding} Layer positions are encoded through learned embeddings $\mathbf{W}_L \in \mathbb{R}^{N_L \times d_{\text{model}}}$ where $N_L$ represents the maximum number of MoE layers in the target architecture. Unlike sinusoidal position encodings used for token sequences, layer positions require learned embeddings because: (1) The hierarchical nature of transformer processing creates non-uniform relationships between layer positions, (2) Different MoE architectures have varying numbers of layers with different routing characteristics, and (3) Layer-specific routing patterns must be learned from data rather than assumed to follow periodic structures.

The layer embeddings enable the model to understand that routing decisions at layer 4 have different semantic implications than identical routing decisions at layer 18. This positional awareness is crucial for accurate prediction, as our analysis shows that the same expert selection can have completely different meanings depending on the layer depth.

\paragraph{Hidden State Integration} Current hidden states $H_t^{(\ell)} \in \mathbb{R}^{d_{\text{model}}}$ from the MoE model are integrated through a learned linear projection $\mathbf{W}_H \in \mathbb{R}^{d_{\text{model}} \times d_{\text{model}}}$ that maps the MoE hidden dimension to the predictor dimension. This projection serves multiple purposes: (1) Dimensional alignment between potentially different model architectures, (2) Feature selection to extract routing-relevant information from high-dimensional hidden states, and (3) Learned transformation that emphasizes routing-predictive features while de-emphasizing irrelevant information.

\subsubsection{Cross-Layer Attention Mechanism}

A key architectural innovation is our cross-layer attention mechanism that explicitly models dependencies between routing decisions across different transformer layers. Standard self-attention in transformers operates within sequences of tokens; our cross-layer attention operates across sequences of layers, capturing the hierarchical routing progression inherent in MoE architectures.

The cross-layer attention mechanism processes the concatenated input representation:
\begin{align}
\mathbf{X}_t = \text{Concat}(&[\mathbf{W}_E[e_t^{(\ell-c+1)}] + \mathbf{W}_L[\ell-c+1], \ldots, \\
&\mathbf{W}_E[e_t^{(\ell)}] + \mathbf{W}_L[\ell], \mathbf{W}_H \cdot H_t^{(\ell)}])
\end{align}

The attention mechanism computes weighted combinations of routing history based on predictive relevance:
\begin{align}
\text{CrossLayerAttn}(\mathbf{X}_t) &= \text{Attention}(\mathbf{Q}, \mathbf{K}, \mathbf{V}) \\
&= \text{softmax}\left(\frac{\mathbf{Q}\mathbf{K}^T}{\sqrt{d_{\text{head}}}}\right)\mathbf{V}
\end{align}

where $\mathbf{Q}$, $\mathbf{K}$, and $\mathbf{V}$ are query, key, and value matrices derived from $\mathbf{X}_t$. This mechanism enables the predictor to learn that routing patterns from certain layers are more predictive for future expert selections than others.

\subsection{Training Methodology}% and Data Engineering}
\label{subsec:training_methodology}

\subsubsection{Dataset Construction and Trace Collection}

Training data consists of expert routing traces collected from pre-trained MoE models processing diverse linguistic tasks. Our dataset construction methodology ensures robust generalization across different domains and routing patterns:

\paragraph{Dataset Diversity} We collect traces from three complementary datasets: (1) Natural Questions (334k traces) for factual question-answering, demonstrating structured reasoning patterns with predictable expert transitions, (2) IMDB movie reviews (125k traces) for sentiment analysis, capturing emotional and stylistic routing patterns with subjective expert selections, and (3) CNN/DailyMail (89k traces) for abstractive summarization, exhibiting complex semantic progressions from content understanding to generation.

\paragraph{Trace Collection Protocol} For each input sequence, we record: expert selections $E_t^{(\ell)}$ at every MoE layer $\ell$ for each token position $t$, hidden states $H_t^{(\ell)}$ immediately before routing decisions, attention masks and token position metadata, routing confidence scores from the original routers, and timing information for expert loading and execution.

\paragraph{Data Quality Assurance} Collected traces undergo quality filtering to remove: sequences shorter than 10 tokens (insufficient context for pattern learning), sequences with excessive routing failures or timeouts, traces with corrupted expert selections or missing hidden states, and outlier sequences with routing patterns more than 3 standard deviations from the mean.

\subsubsection{Training Objective and Loss Function Design}

Our composite loss function balances multiple training objectives to produce well-calibrated predictions suitable for production deployment:

\paragraph{Primary Classification Loss} The cross-entropy loss optimizes expert prediction accuracy:
\begin{align}
\mathcal{L}_{\text{classification}} = -\sum_{i=1}^{N} \log P(e_i^{\text{true}} | \mathcal{H}_i; \theta)
\end{align}
where $N$ is the number of training examples, $e_i^{\text{true}}$ is the ground-truth expert selection, and $P(\cdot | \mathcal{H}_i; \theta)$ represents the predicted probability distribution.

\paragraph{Ranking Consistency Loss} The ranking loss ensures proper ordering of expert selection probabilities:
\begin{align}
\mathcal{L}_{\text{ranking}} = \sum_{i=1}^{N} \sum_{j \neq e_i^{\text{true}}} \max(0, margin - \log P(e_i^{\text{true}} | \mathcal{H}_i) + \log P(j | \mathcal{H}_i))
\end{align}
with margin $= 0.1$. This loss encourages the model to not only predict the correct expert but also maintain meaningful probability rankings for alternative experts, which is crucial for confidence estimation.

\paragraph{Confidence Calibration Loss} The Brier score ensures prediction confidence correlates with actual accuracy:
\begin{align}
\mathcal{L}_{\text{confidence}} = \sum_{i=1}^{N} (P(e_i^{\text{true}} | \mathcal{H}_i) - \mathbb m{1}_{e_i^{\text{true}}})^2
\end{align}
where $\mathbb m{1}_{e_i^{\text{true}}}$ is the indicator function. Well-calibrated confidence scores enable dynamic prefetching strategies that adapt resource allocation based on prediction certainty.

\subsubsection{Hyperparameter Optimization and Training}% Protocol}

\paragraph{Architecture Hyperparameters} Context window size $c$ was determined through systematic evaluation over $c \in \{1, 2, 3, 4, 5\}$, revealing optimal performance at $c = 3$ layers. Shorter contexts ($c < 3$) lack sufficient routing history for pattern recognition, while longer contexts ($c > 3$) introduce noise and computational overhead without accuracy improvements.

Prediction horizon $h$ was optimized across $h \in \{1, 2, 3, 4, 5, 6\}$ with different optima for different architectures: $h = 2$ for Switch Transformer (88.5\% accuracy), $h = 3$ for Qwen1.5 MoE (85\% accuracy), and $h = 3$ for Qwen3 MoE (82\% accuracy). These architecture-specific optima reflect different routing complexity and information retention characteristics.

\paragraph{Training Configuration} We employ AdamW optimization with learning rate $\eta = 2 \times 10^{-4}$, $\beta_1 = 0.9$, $\beta_2 = 0.999$, weight decay $\lambda = 0.01$, and gradient clipping at norm 1.0. The learning rate schedule combines linear warmup over 500 steps (approximately 3\% of total training) followed by cosine annealing to $10^{-6}$.

Training proceeds for 15,000 steps with batch size 128, requiring 3.5-4.2 hours on NVIDIA A100 GPUs depending on the target MoE architecture. Loss weights are set to $\lambda_1 = 1.0$ (classification), $\lambda_2 = 0.3$ (ranking), and $\lambda_3 = 0.1$ (confidence) based on validation performance across multiple architectures.

\subsection{Prediction Horizon Analysis and Architecture-Specific Optimization}
\label{subsec:horizon_analysis}

\subsubsection{Information Decay and Horizon Selection}

Our analysis reveals that prediction accuracy follows an exponential decay pattern with increasing prediction horizon, consistent with information-theoretic expectations. \Cref{fig:predictor_accuracy,fig:qwen15_prediction_accuracy,fig:qwen3_prediction_accuracy} demonstrate this decay across all evaluated architectures.
\begin{figure}[t]
\centering
\includegraphics[width=\linewidth]{figs/qwen15_prediction_accuracy.pdf}
\caption{Predictor accuracy vs batch size for Switch transformer against the state-of-the-art. We observe a rapid decline is the accuracy beyond the prediction horizon, h = 3.}
\label{fig:qwen15_prediction_accuracy}
\end{figure}

\begin{figure}[t]
\centering
\includegraphics[width=\linewidth]{figs/qwen3_prediction_accuracy.pdf}
\caption{Predictor accuracy vs batch size for Switch transformer against the state-of-the-art. We observe a rapid decline is the accuracy beyond the prediction horizon, h = 3.}
\label{fig:qwen3_prediction_accuracy}
\end{figure}

For Switch Transformer, accuracy decreases from 90\% at $h=1$ to 88.5\% at $h=2$, then exhibits steeper decline beyond $h=3$, reaching 20\% at $h=10$. The relatively flat region from $h=1$ to $h=3$ suggests that temporal correlations in Switch Transformer routing remain strong across 3 layers, likely due to the sparse top-1 expert selection creating strong routing consistency.

Qwen MoE architectures show different patterns: both Qwen1.5 and Qwen3 exhibit optimal performance at $h=3$ (85\% and 82\% respectively) with more gradual initial decay than Switch Transformer. This difference stems from their dense top-k routing ($k=4$ for Qwen1.5, $k=8$ for Qwen3) which creates more complex routing patterns that require longer context windows for accurate prediction but provide more stable prediction accuracy across moderate horizons.

\subsubsection{Architecture-Specific Design Adaptations}

\paragraph{Switch Transformer Adaptations} Switch Transformer's sparse top-1 routing creates unique challenges and opportunities: (1) High prediction accuracy is achievable due to deterministic single-expert selection, but (2) Prediction errors have severe consequences since only one expert is selected per token. Our predictor addresses this through aggressive confidence thresholding, only prefetching when confidence exceeds 80\%.

\paragraph{Qwen MoE Adaptations} Dense routing in Qwen architectures (top-4 and top-8 selection) requires different prediction strategies: (1) Top-k prediction instead of single expert prediction, providing multiple prefetching candidates, (2) Weighted prefetching based on predicted routing probabilities, and (3) Shared expert handling for experts that process every token regardless of routing decisions.

\subsubsection{Confidence Calibration and Dynamic Prefetching}

The predictor outputs calibrated confidence scores alongside expert predictions, enabling dynamic prefetching strategies. Confidence calibration is achieved through the Brier score component of our loss function and validated through reliability diagrams showing strong correlation between predicted confidence and actual accuracy.

High-confidence predictions ($> 0.8$) trigger immediate prefetching to L1 cache for sub-millisecond access, medium-confidence predictions (0.5-0.8) initiate L2 cache prefetching for moderate latency access, and low-confidence predictions either prefetch to L3 cache or are deferred based on available memory bandwidth.

This confidence-driven approach ensures optimal resource allocation: limited prefetching bandwidth is reserved for predictions most likely to be correct, maximizing cache hit rates while minimizing waste on incorrect predictions.

\subsection{Computational Efficiency and Production Deployment}
\label{subsec:computational_efficiency}

\subsubsection{Inference Overhead Analysis}

The neural predictor introduces minimal computational overhead relative to MoE inference. With 8.4M parameters compared to 1.6B+ parameters in typical MoE models (0.5\% parameter overhead), prediction requires 0.8-1.2ms compared to 15-50ms for MoE forward passes (2-8\% latency overhead). Memory overhead is similarly minimal: 67MB for predictor weights compared to 3.2-12.8GB for MoE parameters (0.5-2\% memory overhead).

\subsubsection{Scalability and Deployment Considerations}

The predictor architecture scales efficiently across different MoE configurations through: (1) Modular embedding matrices that adapt to different expert vocabulary sizes, (2) Architecture-agnostic hidden state projection layers, and (3) Configurable context windows and prediction horizons optimized per deployment scenario.

Production deployment involves loading predictor checkpoints alongside MoE models, minimal memory allocation for inference state, and integration with existing MoE serving infrastructure through standardized APIs. The predictor runs asynchronously with MoE inference, enabling pipelined execution where predictions for layer $\ell+h$ are computed while processing layer $\ell$.



% \begin{table}[t]
% \centering
% \begin{tabular}{lcc}
% \toprule
% \textbf{Configuration} & \textbf{Switch-T} & \textbf{Qwen} \\
% \midrule
% Model Parameters & 8.4M & 8.4M \\
% Context Window ($c$) & 3 layers & 3 layers \\
% Prediction Horizon ($h$) & 3 layers & 3 layers \\
% Training Time & 3.5 hours & 4.2 hours \\
% \midrule
% Accuracy & 86.42\% & 81.35\% \\
% Random Baseline & 0.78\% & 1.56\% \\
% Improvement Factor & $110.79\times$ & $52.14\times$ \\
% \midrule
% Inference Latency & 0.15ms & 0.18ms \\
% Memory Overhead & 0.32\% & 0.06\% \\
% \bottomrule
% \end{tabular}
% \caption{Neural predictor performance across architectures.}
% \label{tab:predictor-performance}
% \end{table}

\subsection{Detailed Accuracy Metrics}
\label{subsec:accuracy_metrics}

\Cref{tab:accuracy_detailed} provides a comprehensive breakdown of prediction accuracy metrics across all evaluated models. We report both Top-1 accuracy (whether the single highest-probability prediction matches ground truth) and Recall@k (whether all k router-selected experts appear in the top-k predictions). Top-1 is a stricter metric that measures precise prediction; Recall@k measures effective prefetching coverage.

\begin{table}[h]
\centering
\small
\begin{tabular}{lcccc}
\toprule
\textbf{Model} & \textbf{Top-1 Acc.} & \textbf{Recall@k} & \textbf{k} & \textbf{Random} \\
\midrule
Qwen1.5-MoE (M1) & 80.4\% & 94.2\% & 4 & 6.7\% \\
DeepSeek (M2) & 88.6\% & 96.8\% & 6 & 9.4\% \\
GPT-OSS (M3) & 86.6\% & 95.1\% & 4 & 12.5\% \\
Qwen3 (M4) & 80.4\% & 93.7\% & 8 & 6.3\% \\
Mixtral (M5) & 82.4\% & 91.3\% & 2 & 25.0\% \\
\bottomrule
\end{tabular}
\caption{Prediction accuracy breakdown. Top-1 measures if the single highest-probability prediction is correct; Recall@k measures if all router-selected experts are in Prophet's top-k predictions. Random baseline = k/N where N is total experts. All models substantially exceed random baselines by 6--15$\times$.}
\label{tab:accuracy_detailed}
\end{table}

Prophet achieves 80--89\% Top-1 accuracy across all models, with Recall@k exceeding 91\% in all cases. The higher Recall@k values (compared to Top-1) indicate that Prophet's probability ranking is well-calibrated: even when the top-1 prediction is incorrect, the true experts typically appear in the top-k predictions. This property is crucial for effective prefetching, where coverage matters more than strict precision.

\subsection{Systematic Comparison with SoTA}
\label{subsec:related_comparison}

\Cref{tab:related_comparison} provides a systematic comparison of Prophet with prior expert prediction approaches across key dimensions: prediction method, training requirements, context window (number of previous layers used), cross-domain transfer evaluation, and number of models evaluated.

\begin{table*}[h]
\centering
\small
\begin{tabular}{lccccc}
\toprule
\textbf{System} & \textbf{Method} & \textbf{Train} & \textbf{Context} & \textbf{Cross-} & \textbf{Models} \\
& & \textbf{Req'd} & \textbf{Layers} & \textbf{Domain} & \textbf{Eval'd} \\
\midrule
ProMoE~\cite{promoe} & LRU stride & No & 0 & N/R & 2 \\
PreGated~\cite{pregated} & Router mod & Yes$^*$ & 1 & N/R & 1 \\
FATE~\cite{fate} & Gate lookahead & No & 1 & N/R & 2 \\
AdapMoE~\cite{adapmoe} & Single lookahead & No & 1 & N/R & 2 \\
PreScope~\cite{prescope} & Layer-group MLP & Yes & 1 & N/R & 2 \\
DuoServe~\cite{duoserve} & Decode lookahead & Yes & 1 & N/R & 3 \\
\midrule
\textbf{Prophet} & Neural transformer & Yes & 3 & 67--98\% & 5 \\
\bottomrule
\end{tabular}
\caption{Comparison of expert prediction approaches. $^*$PreGated requires model retraining. N/R = Not Reported. Prophet's information-theoretic foundation, 3-layer context window, and cross-domain evaluation differentiate it from prior work.}
\label{tab:related_comparison}
\end{table*}

Key differentiators include: (1) Prophet uses a 3-layer context window compared to 0--1 layers in prior work, enabling capture of cross-layer routing dependencies; (2) Prophet is the only approach to systematically evaluate cross-domain transfer, demonstrating 67--98\% accuracy across all domain pairs; (3) Prophet evaluates on 5 diverse MoE architectures, compared to 1--3 in prior work.
\section{Memory Management and Caching}
\label{app:memoryCaching}

\begin{algorithm}[h!]
\caption{Batch-Aware Expert Deduplication}
\label{alg:deduplication}
\begin{algorithmic}[1]
\Require Batch $\mathcal{B} = \{R_1, \ldots, R_B\}$, $R_i$ = experts for item $i$
\Ensure Unique set $\mathcal{U}$, schedule $\mathcal{S}$
\State $\mathcal{U} \gets \emptyset$; $\mathcal{F} \gets \{\}$ \Comment{Init unique set, freq map}
\For{each request $R_i \in \mathcal{B}$}
    \For{each expert $e \in R_i$}
        \State $\mathcal{U} \gets \mathcal{U} \cup \{e\}$
        \State $\mathcal{F}[e] \gets \mathcal{F}[e] + 1$
    \EndFor
\EndFor
\State Sort $\mathcal{U}$ by frequency $\mathcal{F}$ descending
\State $\mathcal{S} \gets$ ComputeOptimalSchedule($\mathcal{U}$, $\mathcal{F}$)
\State \Return $\mathcal{U}$, $\mathcal{S}$
\end{algorithmic}
\end{algorithm}


\subsection{Hardware-Aware Transfer Optimization}
The two-level cache design enables sophisticated transfer optimization that hides PCIe latency behind computation. When high-confidence predictions identify experts currently in L2  cache, the system initiates asynchronous PCIe transfers to promote these experts to L1 cache while the current layer continues computation. This reduces effective loading latency by overlapping memory transfers with useful computation
:
%\begin{equation}
$T_{\text{effective}} = \max(T_{\text{compute}}, T_{\text{PCIe\_transfer}}) + T_{\text{GPU\_access}}$
%\end{equation}
where $T_{\text{GPU\_access}} \ll T_{\text{PCIe\_transfer}}$, 
enabling near-zero effective latency when transfers complete before expert usage.
The transfer scheduler incorporates PCIe bandwidth constraints and GPU memory pressure to optimize transfer ordering. High-confidence predictions receive priority for immediate transfer, while medium-confidence predictions are queued based on available bandwidth. The scheduler dynamically adjusts the confidence threshold for GPU cache placement based on current cache occupancy and predicted expert usage patterns, ensuring optimal resource utilization under varying workload conditions.

\subsection{Cache Performance Characteristics}

The two-level cache design achieves superior hit rates compared to traditional single-tier approaches by exploiting the natural memory hierarchy. L1 GPU cache hit rates of 85-90\% ensure that the majority of expert accesses incur minimal latency, while L2 host cache hit rates of 95-99\% provide comprehensive coverage for the remaining expert requests. The combined system achieves overall hit rates of 99.4\% under iso-cache constraints, representing significant improvements over baseline LRU caching (65-80\% hit rates).

\Cref{tab:cache_stats} provides per-model cache statistics validating the $\sim$98\% effective hit rate claim from the main text. The breakdown shows that L1 hits dominate (81--86\%) due to high prediction accuracy and power-law caching of popular experts, while L2 provides speculative coverage for medium-confidence predictions (12--17\%), with cold misses limited to $\sim$2\%.

\begin{table}[h]
\centering
\small
\begin{tabular}{lcccc}
\toprule
\textbf{Model} & \textbf{L1 Hit} & \textbf{L2 Hit} & \textbf{Cold Miss} & \textbf{Total Hit} \\
\midrule
Qwen1.5-MoE (M1) & 82.3\% & 15.7\% & 2.0\% & 98.0\% \\
DeepSeek (M2) & 86.4\% & 12.1\% & 1.5\% & 98.5\% \\
GPT-OSS (M3) & 84.8\% & 13.0\% & 2.2\% & 97.8\% \\
Qwen3 (M4) & 81.2\% & 16.5\% & 2.3\% & 97.7\% \\
Mixtral (M5) & 83.6\% & 14.2\% & 2.2\% & 97.8\% \\
\bottomrule
\end{tabular}
\caption{Cache hit rate breakdown per model. L1 hits dominate due to high prediction accuracy and power-law caching of popular experts; L2 provides speculative coverage for medium-confidence predictions; cold misses are limited to $\sim$2\%.}
\label{tab:cache_stats}
\end{table}

Cache performance scales favorably with batch size due to the expert deduplication benefits identified earlier. Larger batches increase the probability that multiple tokens will request the same experts, improving effective cache utilization and reducing the pressure on both cache tiers. The power-law popularity distribution ensures that frequently accessed experts remain resident in L1 GPU cache across multiple batches, while less common experts benefit from L2 host cache coverage.

\subsection{Transfer-Compute Overlap Analysis}

\Cref{tab:transfer_overlap} quantifies whether the prediction horizon provides sufficient lead time to hide expert transfers across PCIe. For models M1, M2, and M4, the overlap factor exceeds 1.0$\times$, enabling full transfer hiding---the 3-layer prediction horizon provides more lead time than required for transfer completion. For M3 (GPT-OSS) and M5 (Mixtral) with larger experts, partial overlap (1.1--1.3$\times$) combines with high L1 cache hit rates (83--85\%, \Cref{tab:cache_stats}) to achieve observed speedups: most experts are already GPU-resident, so transfers occur infrequently. Mixtral's 8.0$\times$ speedup despite 1.1$\times$ overlap demonstrates that caching and prediction synergistically reduce effective transfer frequency.

\begin{table}[h]
\centering
\small
\begin{tabular}{lccccc}
\toprule
\textbf{Model} & \textbf{Expert} & \textbf{Transfer} & \textbf{Compute} & \textbf{Lead} & \textbf{Overlap} \\
& \textbf{Size} & \textbf{Time} & \textbf{/Layer} & \textbf{Time} & \textbf{Factor} \\
\midrule
M1 (Qwen1.5) & 100MB & 3.1ms & 4.2ms & 12.6ms & 4.1$\times$ \\
M2 (DeepSeek) & 250MB & 7.8ms & 5.1ms & 15.3ms & 2.0$\times$ \\
M3 (GPT-OSS) & 600MB & 18.8ms & 8.3ms & 24.9ms & 1.3$\times$ \\
M4 (Qwen3) & 250MB & 7.8ms & 7.2ms & 21.6ms & 2.8$\times$ \\
M5 (Mixtral) & 1.3GB & 40.6ms & 15.0ms & 45.0ms & 1.1$\times$ \\
\bottomrule
\end{tabular}
\caption{Transfer-compute overlap analysis at PCIe 4.0 (32GB/s). Transfer Time = Expert Size / Bandwidth. Lead Time = 3 $\times$ Compute/Layer (h=3 horizon). Overlap Factor = Lead Time / Transfer Time; values $\geq$1 indicate full hiding capability.}
\label{tab:transfer_overlap}
\end{table}

\subsection{Integration with Neural Prediction and Batch Optimization}
The two-level cache seamlessly integrates with the neural predictor and batch deduplication components to form a cohesive expert prefetching system. Prediction confidence scores directly drive cache placement decisions, ensuring that GPU memory is reserved for the most reliable predictions while host memory provides speculative coverage. Batch deduplication reduces the total number of unique experts that must be managed across both cache levels, improving effective cache capacity and reducing transfer overhead.
The cache system adapts automatically to different MoE architectures by adjusting confidence thresholds and promotion criteria. Sparse routing models (Switch Transformer) benefit from aggressive L1 caching of high-confidence predictions due to lower expert diversity per batch, while dense routing models (Qwen MoE) leverage L2 cache more extensively to accommodate the larger set of potentially required experts. 
%This architectural adaptation ensures optimal performance across diverse MoE configurations without manual tuning.

\subsection{Formalization of Promotion Mechanism}
The promotion mechanism operates according to:
\begin{equation}
\begin{aligned}
\text{Promote}(e, L_2^{\text{Host}} \to L_1^{\text{GPU}}) \\
\text{if} \quad \mathcal{C}[e] > \theta_{\text{promote}} 
   \ \text{or } \text{AccessFreq}(e) > \theta_{\text{freq}}
\end{aligned}
\end{equation}
where $\theta_{\text{promote}} = 0.8$ ensures high-confidence predictions trigger immediate promotion, and $\theta_{\text{freq}}$ captures experts that demonstrate sustained usage patterns even with lower individual prediction confidence.
\section{Mixture of Experts Architectures: A Comprehensive Analysis}
\label{appedndix:MoEModels}

\subsection{Architectural Patterns in MoE Models}

Mixture of Experts (MoE) models have emerged as a dominant paradigm for scaling language models beyond traditional dense architectures. These models achieve computational efficiency through conditional computation, activating only a subset of parameters for each input token. We identify several key architectural patterns across modern MoE implementations:

\subsubsection{Routing Mechanisms}

The routing mechanism determines which experts process each token. Three primary strategies have emerged:

\textbf{Top-k Routing:} The predominant approach, where k experts are selected based on highest routing scores. Top-2 routing (Mixtral, Grok-1, Arctic, GShard) represents the industry standard, balancing computational efficiency with model capacity. Higher k values (DeepSeek's top-6 to top-12) enable finer-grained specialization but increase computational overhead.

\textbf{Capacity-Constrained Routing:} Implemented in models like GShard and Switch Transformer, this approach limits the number of tokens each expert can process, preventing load imbalance but potentially dropping tokens when capacity is exceeded.

\textbf{Threshold-Based Routing:} Used in ST-MoE variants, where experts are activated only when routing confidence exceeds a learned threshold, enabling dynamic computation allocation.

\subsubsection{Expert Architecture Strategies}

\textbf{Homogeneous Experts:} Most models employ identically-structured experts (Mixtral, Grok-1), simplifying implementation and load balancing. Expert sizes range from tiny (29MB in DeepSeekMoE-16B) to massive (38GB in Grok-1).

\textbf{Shared Experts:} A hybrid approach pioneered by DeepSeek and Qwen, where 1-4 experts process all tokens while remaining experts are selectively routed. This ensures baseline model capacity while maintaining specialization benefits. DeepSeek-V3's configuration (1 shared + 256 routed) exemplifies this pattern at scale.

\textbf{Heterogeneous Experts:} Emerging in models like ERNIE 4.5, where experts vary in size and capability, potentially matching computational resources to task complexity.

\subsubsection{Scaling Strategies}

Two distinct scaling philosophies have emerged:

\textbf{Few Large Experts:} Models like Mixtral-8x7B and Grok-1 use 8 experts with billions of parameters each, maximizing individual expert capacity while minimizing routing overhead.

\textbf{Many Small Experts:} GShard (2048 experts) and DeepSeek-V3 (257 experts) employ numerous smaller experts, enabling fine-grained specialization at the cost of increased routing complexity.

\subsection{Activation Density and Efficiency}

Activation density—the percentage of parameters used per forward pass—critically impacts inference efficiency. Low activation density (0.78\% in Switch Transformer) maximizes parameter efficiency but requires sophisticated routing. Higher density (27.7\% in Mixtral-8x22B) simplifies deployment but reduces the efficiency gains from sparse activation.

The relationship between expert count and activation density follows an inverse power law: doubling experts typically halves activation density, assuming constant active expert count. This trade-off fundamentally constrains MoE architecture design, balancing model capacity, inference speed, and memory requirements.

%%%%%%%%%%%%%%%%%%%%%%%%%%%%%%%%%%%%%
\begin{table*}[t]
\centering
\begin{tabularx}{\linewidth}{@{}lXXXXX@{}}
\toprule
\textbf{Model} & \textbf{Experts} & \textbf{Routing} & \textbf{Expert Size} & \textbf{Activation Density} & \textbf{Total Parameters} \\
\midrule
\multicolumn{6}{l}{\textit{Early MoE Models}} \\
Switch Transformer & 128 & top-1 & 6.3MB & 0.78\% & 1.6T \\
GShard & 2048 & top-2 & 293MB & 0.10\% & 600B \\
GLaM-64B & 64 & top-2 & 1GB & 3.13\% & 1.2T \\
ST-MoE-32B & 32 & top-2 & 8.4GB & 6.25\% & 269B \\
\midrule
\multicolumn{6}{l}{\textit{Mixtral Family (Mistral AI)}} \\
Mixtral-8x7B & 8 & top-2 & 6.45GB & 27.6\% & 46.7B \\
Mixtral-8x22B & 8 & top-2 & 19.5GB & 27.7\% & 141B \\
\midrule
\multicolumn{6}{l}{\textit{DeepSeek Models}} \\
DeepSeekMoE-16B & 65 (1+64) & top-6 & 29MB & 17.1\% & 16.4B \\
DeepSeekMoE-145B & 132 (4+128) & top-12 & 1.1GB & 15.4\% & 144.6B \\
DeepSeek-V2 & 162 (2+160) & top-6 & 1.46GB & 8.9\% & 236B \\
DeepSeek-V3 & 257 (1+256) & top-8 & 2.61GB & 5.5\% & 671B \\
\midrule
\multicolumn{6}{l}{\textit{Industry Models}} \\
Snowflake Arctic & 128 & top-2 & 3.66GB & 3.5\% & 480B \\
Grok-1 (xAI) & 8 & top-2 & 38.25GB & 25.0\% & 314B \\
Qwen1.5-MoE-A2.7B & 64 (4+60) & top-4 & 386MB & 12.5\% & 14.3B \\
PaLM-540B MoE & 32 & top-2 & 13GB & 6.25\% & 540B \\
\midrule
\multicolumn{6}{l}{\textit{Specialized MoE Models}} \\
NLLB-MoE (Meta) & 128 & top-2 & 421MB & 25.0\% & 54B \\
Jamba (AI21) & 16 & top-2 & 3.25GB & 23.0\% & 52B \\
ERNIE 4.5 (Baidu) & 64 & top-2 & 6.6GB & 11.0\% & 424B \\
OpenMoE-8B & 32 & top-2 & 250MB & 6.25\% & 8B \\
\bottomrule
\end{tabularx}
\caption{Comprehensive Comparison of Mixture of Experts Model Architectures}
\label{tab:moe_models_comprehensive}
\end{table*}
%%%%%%%%%%%%%%%%%%%%%%%%%%%%%%%%%%%%%

\subsection{Memory and Computational Requirements}

The memory footprint of MoE models presents unique deployment challenges. While only a fraction of parameters are active during inference, the entire model must reside in memory. For instance, Mixtral-8x7B requires approximately 87GB of VRAM despite activating only 12.9B parameters, as all 46.7B parameters must be accessible for routing decisions.

This memory-computation asymmetry fundamentally distinguishes MoE from dense models: they offer superior compute efficiency (FLOPs per token) but require substantially more memory per active parameter. Recent architectures like DeepSeek-V3 push this trade-off to extremes, achieving 5.5\% activation density—processing each token with just 37B of 671B total parameters.

\subsection{Evolution and Future Directions}

The progression from Switch Transformer's 1.6T parameters with 0.78\% activation to DeepSeek-V3's 671B parameters with 5.5\% activation reveals an industry shift toward more practical deployment scenarios. Modern architectures increasingly favor:

\begin{itemize}
\item \textbf{Moderate expert counts} (8-256) over extreme scaling (2048 experts in GShard)
\item \textbf{Shared expert baselines} ensuring consistent performance across all inputs
\item \textbf{Higher routing degrees} (top-4 to top-12) for improved load balancing
\item \textbf{Adaptive routing mechanisms} that adjust computation based on input complexity
\end{itemize}

These trends suggest MoE architectures are maturing from research curiosities to production-ready systems, balancing theoretical efficiency gains with practical deployment constraints.

\section{Throughput Evaluation Details}
\label{sec:throughput-details}

This appendix provides detailed analysis of Prophet's throughput-critical performance, including mathematical derivation of deduplication benefits from power-law distributions, comprehensive batch scaling results, and multi-GPU tensor parallelism evaluation.

\subsection{Mathematical Foundation: Power-Law to Deduplication}
\label{subsec:throughput-details}

The power-law expert popularity distribution ($\alpha = 1.2$--$2.2$) directly enables bandwidth savings through batch-level deduplication. We derive the expected number of unique experts $U(B)$ in a batch of size $B$.

For a power-law distribution with exponent $\alpha$, expert $i$ has selection probability $p_i \propto i^{-\alpha}$. The probability that expert $i$ is \textit{not} selected by any of $B$ independent tokens is $(1-p_i)^B$. Thus, the expected number of unique experts is:
\begin{equation}
\mathbb{E}[U(B)] = \sum_{i=1}^{N} \left(1 - (1-p_i)^B\right) \approx N \cdot \left(1 - e^{-B/N_{\text{eff}}}\right)
\end{equation}
where $N_{\text{eff}}$ is the effective expert count accounting for power-law concentration. Empirical fitting across our five models yields:
\begin{equation}
\mathbb{E}[U(B)] \sim B^{0.932}
\end{equation}
This sub-linear scaling (exponent $<1$) quantifies deduplication: at $B=16$, we expect $\sim$13.4 unique experts versus 16 with uniform distribution---a 16\% bandwidth reduction from deduplication alone.

\noindent\textbf{Intuition.} The power-law concentration means a small number of ``popular'' experts handle most routing decisions. When batching multiple tokens, these popular experts are likely requested by multiple tokens simultaneously. Rather than fetching duplicate copies, Prophet identifies unique experts across the batch and fetches each only once. The stronger the power-law (higher $\alpha$), the greater the deduplication benefit.

\subsection{Batch Size Scaling Results}

\Cref{tab:batch-detailed} presents Prophet's speedup across batch sizes for all five evaluated models. The speedup remains stable with $<$5\% relative decrease at larger batches.

\begin{table}[h]
\centering
\small
\begin{tabular}{lcccc}
\toprule
\textbf{Model} & \textbf{BS=1} & \textbf{BS=4} & \textbf{BS=8} & \textbf{BS=16} \\
\midrule
GPT-OSS-20B & 10.38$\times$ & 10.10$\times$ & 10.00$\times$ & 9.90$\times$ \\
Mixtral-8x7B & 8.03$\times$ & 7.70$\times$ & 7.60$\times$ & 7.50$\times$ \\
DeepSeek-MoE-16B & 3.64$\times$ & 3.54$\times$ & 3.47$\times$ & 3.40$\times$ \\
Qwen3-30B & 2.74$\times$ & 2.64$\times$ & 2.57$\times$ & 2.50$\times$ \\
Qwen1.5-MoE & 1.53$\times$ & 1.50$\times$ & 1.47$\times$ & 1.44$\times$ \\
\bottomrule
\end{tabular}
\caption{Detailed speedup vs.\ batch size. Prophet maintains 1.4--10$\times$ speedup across BS=1--16 with stable relative performance.}
\label{tab:batch-detailed}
\end{table}

The speedup stability arises because Prophet's prediction accuracy is per-token (batch-independent), while baselines benefit from incidental expert reuse at larger batches---partially closing the gap but never eliminating Prophet's prediction advantage.

\subsection{Bandwidth Comparison with Baselines}

\Cref{tab:bandwidth-comparison} compares bandwidth requirements across systems at different batch sizes. Prophet's deduplication provides substantial savings over approaches without deduplication, particularly PreScope.

\begin{table}[h]
\centering
\small
\begin{tabular}{lccc}
\toprule
\textbf{System} & \textbf{Experts/Token} & \textbf{Deduplication} & \textbf{Bandwidth @ BS=16} \\
\midrule
On-Demand & $k$ (top-k) & Per-token & $16 \times k$ \\
PreScope & $4$ (top-4) & None & $16 \times 4 = 64$ \\
Prophet & $k$ (top-k) & Batch-level & $\sim 13.4 \times k$ \\
\bottomrule
\end{tabular}
\caption{Bandwidth comparison. PreScope's top-4 prediction without deduplication requires 4$\times$ more bandwidth than Prophet at equivalent batch sizes.}
\label{tab:bandwidth-comparison}
\end{table}

At BS=16, Prophet fetches $\sim$0.84$\times$ the bandwidth of On-Demand (due to deduplication) and $\sim$0.21$\times$ the bandwidth of PreScope (due to both deduplication and top-1 vs.\ top-4 prediction).

\noindent\textbf{Bandwidth Scaling Analysis.} \Cref{tab:bandwidth-scaling} provides detailed bandwidth ratios across batch sizes for all systems.

\begin{table}[h]
\centering
\small
\begin{tabular}{lcccc}
\toprule
\textbf{System Comparison} & \textbf{BS=1} & \textbf{BS=4} & \textbf{BS=8} & \textbf{BS=16} \\
\midrule
Prophet / On-Demand & 1.00 & 0.94 & 0.89 & 0.84 \\
Prophet / PreScope & 0.25 & 0.23 & 0.22 & 0.21 \\
Unique Expert Ratio & 1.00 & 0.94 & 0.89 & 0.84 \\
\bottomrule
\end{tabular}
\caption{Bandwidth ratios across batch sizes. Prophet's advantage over PreScope (0.21--0.25$\times$) combines deduplication benefit with top-1 vs.\ top-4 prediction efficiency.}
\label{tab:bandwidth-scaling}
\end{table}

\subsection{Multi-GPU Tensor Parallelism Results}

\Cref{tab:tp-detailed} presents detailed tensor parallelism scaling results. Prophet's speedup advantage is preserved across TP=1--8, with relative performance remaining within 2\%.

\begin{table}[h]
\centering
\small
\begin{tabular}{lcccc}
\toprule
\textbf{Model} & \textbf{TP=1} & \textbf{TP=2} & \textbf{TP=4} & \textbf{TP=8} \\
\midrule
GPT-OSS-20B & 10.4$\times$ & 10.4$\times$ & 10.3$\times$ & 10.2$\times$ \\
Mixtral-8x7B & 8.0$\times$ & 8.0$\times$ & 7.9$\times$ & 7.8$\times$ \\
DeepSeek-MoE-16B & 3.6$\times$ & 3.6$\times$ & 3.6$\times$ & 3.5$\times$ \\
Qwen3-30B & 2.7$\times$ & 2.7$\times$ & 2.7$\times$ & 2.6$\times$ \\
Qwen1.5-MoE & 1.5$\times$ & 1.5$\times$ & 1.5$\times$ & 1.5$\times$ \\
\bottomrule
\end{tabular}
\caption{Tensor parallelism scaling. Prophet's speedup remains stable across TP=1--8 because routing decisions are determined by model architecture, not parallelization strategy.}
\label{tab:tp-detailed}
\end{table}

In tensor-parallel configurations, expert loading becomes a per-GPU operation with experts partitioned across devices. Prophet's predictions remain accurate because the router weights and routing logic are replicated across all GPUs, producing identical expert selection decisions regardless of how experts are distributed.

\noindent\textbf{Why TP Doesn't Affect Prediction.} The MoE router is a small linear layer ($d_{\text{model}} \times N_{\text{experts}}$) that is replicated across all GPUs in tensor parallelism. Each GPU computes identical routing decisions from identical hidden states. Expert parallelism (EP) partitions experts across GPUs, but Prophet's predictor operates on routing decisions, not expert locations---making predictions hardware-agnostic.

\subsection{Cache Residency at Large Batches}

At batch sizes exceeding 16, the power-law concentration creates high cache residency for popular experts. With the top 20\% of experts handling 75\% of routing decisions, these popular experts remain GPU-resident across batch iterations. In this regime, cache management and eviction policies dominate over prediction accuracy---Prophet's advantage shifts from prediction to efficient cache utilization.

\Cref{tab:cache-residency} shows expected cache hit rates at different batch sizes, assuming an LRU cache sized for 25\% of total experts.

\begin{table}[h]
\centering
\small
\begin{tabular}{lcccc}
\toprule
\textbf{Batch Size} & \textbf{BS=1} & \textbf{BS=8} & \textbf{BS=16} & \textbf{BS=32} \\
\midrule
Cache Hit Rate & 0.75 & 0.82 & 0.87 & 0.91 \\
Prediction Value & High & High & Medium & Low \\
\bottomrule
\end{tabular}
\caption{Cache residency increases with batch size due to power-law concentration. At BS$>$16, prediction accuracy becomes less critical as popular experts remain GPU-resident.}
\label{tab:cache-residency}
\end{table}

\noindent\textbf{Design Implications.} This analysis clarifies Prophet's design target: latency-critical BS=1 inference where prediction accuracy directly determines performance. For throughput-critical large-batch workloads, Prophet provides stable speedups but the relative advantage of accurate prediction diminishes as cache management dominates. Future work on cache-aware prediction could further optimize the large-batch regime.

\subsection{Continuous Batching Considerations}

Prophet's per-token routing history naturally extends to continuous batching frameworks (vLLM, TGI) that dynamically schedule requests:

\begin{itemize}
\item \textbf{Per-request history:} Each request maintains its own routing history, persisting across batch iterations as new tokens are generated.
\item \textbf{Cross-request deduplication:} When multiple requests in a continuous batch require the same expert, Prophet's batch-level deduplication applies across all active requests.
\item \textbf{Iteration-level prediction:} Predictions update as new tokens generate, with the predictor processing routing decisions from all active requests in the iteration.
\end{itemize}

The power-law expert concentration ensures popular experts remain cached across request boundaries, providing natural ``warm start'' benefits for new requests joining an active batch.




%%%%%%%%%% %%%%%%%%%% %%%%%%%%%% %%%%%%%%%%

%%%%%%%%%%%%%%%%%%%%%%%%%%%%%%%%%%%%%%%%%%%%%%%%%%%%%%%%%%%%%%%%%%%%%%%%%%%%%%%%
\end{document}
%%%%%%%%%%%%%%%%%%%%%%%%%%%%%%%%%%%%%%%%%%%%%%%%%%%%%%%%%%%%%%%%%%%%%%%%%%%%%%%%
