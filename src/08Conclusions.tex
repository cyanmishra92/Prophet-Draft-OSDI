\vspace{-10pt}
\section{Conclusions}
\label{sec:conclusion}

% Text matter begin
%%%%%%%%%% %%%%%%%%%% %%%%%%%%%% %%%%%%%%%% 
We addressed the fundamental memory bandwidth bottleneck constraining practical deployment of Mixture-of-Experts (MoE) models. Through information-theoretic analysis, we established that expert routing contains exploitable structure---power-law distributions and cross-layer dependencies providing 0.62 bits of mutual information---enabling accurate neural prediction of future expert requirements.

Building on these insights, \emph{Prophet} combines learned cross-layer prediction (80--89\% accuracy) with confidence-guided hierarchical caching, transforming MoE inference from reactive to predictive memory management. Our evaluation across five production architectures demonstrates $1.5\times$--$10.4\times$ TPOT speedups and $2.4\times$--$19.4\times$ tail latency (P99.9) improvements over on-demand loading, while enabling $4$--$8\times$ longer context lengths within the same memory budget. Critically, prediction accuracy remains 67--98\% across all cross-domain pairs, confirming that Prophet learns model-intrinsic routing patterns rather than dataset-specific features.

These results demonstrate that predictive memory management fundamentally shifts MoE execution from memory-bound to compute-bound, removing the dominant I/O barrier that previously limited deployment. The principles underlying Prophet---information-theoretic analysis guiding predictor design, cross-layer neural prediction, and confidence-aware resource allocation---hold promise for broader classes of sparse architectures where scalability and deployment efficiency are critical.
%%%%%%%%%% %%%%%%%%%% %%%%%%%%%% %%%%%%%%%% 