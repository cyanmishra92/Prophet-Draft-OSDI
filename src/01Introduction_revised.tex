%%%%%%%%%% %%%%%%%%%% %%%%%%%%%% %%%%%%%%%% 
\section{Introduction}
\label{sec:01Introduction}

% Text matter begin
%%%%%%%%%% %%%%%%%%%% %%%%%%%%%% %%%%%%%%%% 
The artificial intelligence landscape has undergone a seismic transformation with the emergence of Mixture-of-Experts (MoE) architectures as the dominant paradigm for scaling language models to unprecedented levels. The global market for AI language models, valued at \$8.9 billion in 2025, is projected to reach \$35.4 billion by 2030~\cite{}, driven primarily by the computational efficiency breakthroughs enabled by MoE architectures~\cite{grandview2025}. Major technology companies have invested \$33.9 billion in generative AI development in 2024 alone, with industry leaders deploying trillion-parameter MoE models in production: Meta's Llama 4 Maverick (400B total parameters, 17B active), Google's Gemini 2.5 series, DeepSeek-R1 (671B parameters) and reportedly OpenAI's GPT-4 (1.8T parameters)~\cite{meta2025llama4,deepseek2025}. These deployments have catalyzed a remarkable $280\times$ reduction in inference costs, from \$20 per million tokens in November 2022 to \$0.07 in October 2024, essentially allowing masses to access advanced AI capabilities~\cite{stanford2025ai}.

The economic imperative driving MoE adoption is compelling. Google's Switch Transformer~\cite{fedus2022switch}, with 1.6 trillion parameters, achieves the performance of massive dense models while requiring only the computational resources of a 100+ billion parameter model during training and inference. DeepSeek-R1~\cite{} demonstrates $30\times$ cost efficiency compared to OpenAI's o1 model~\cite{} while maintaining comparable performance. Similarly, Microsoft's DeepSpeed-MoE~\cite{} achieves $7.3\times$ latency reduction and up to $9\times$ cost savings compared to iso-quality dense models~\cite{microsoft2025deepspeed}. However, this efficiency comes with a critical caveat: despite activating only 1-10\% of their massive parameter space, MoE models face severe memory bandwidth bottlenecks, as explained below, that fundamentally limit their deployment potential.

\noindent\textbf{The MoE Efficiency Paradox:}
MoE architectures achieve their remarkable efficiency through ``sparse activation'', where each input token is routed to a small subset of specialized expert networks rather than the entire model, as depicted in \Cref{fig:moe_architecture}. In Switch Transformer's top-1 routing for example, each token activates only 1 out of 128 experts (0.78\% density), while Qwen3-30B-A3B
%Qwen-1.5-MoE 
employs top-8 routing to activate 8 out of 128 experts (6.25\% density) per token. This sparse activation pattern enables models to scale to trillions of parameters while maintaining reasonable computational requirements, the key insight that has driven widespread industry adoption.

However, this computational efficiency masks a fundamental resource paradox. Although MoE models require minimal computation per token, they demand enormous memory capacity to store all experts and suffer from severe memory bandwidth limitations during expert loading. Switch Transformer's 128 experts require approximately 806MB per MoE layer, while Qwen-1.5-MoE demands roughly 23GB for its 60 routing experts, far exceeding even high-end data center GPUs like the NVIDIA A100 with 80GB HBM. More critically, the dynamic nature of expert routing creates unpredictable memory access patterns that defeat traditional caching strategies, forcing systems to load experts on-demand with devastating performance implications.

Our empirical analysis reveals that expert loading dominates 87-95\% of total inference time across different batch sizes and architectures. Transferring a 6.3MB Switch Transformer expert over PCIe 4.0 requires $197\mu$s while actual expert computation takes only $15\mu$s, creating a $13\times$ I/O-to-compute bottleneck. For Qwen-1.5-MoE, loading a 386MB expert requires 12ms while computation requires only 0.8ms, resulting in a devastating $15\times$ bottleneck. This memory bandwidth crisis represents the primary obstacle that prevents MoE models from realizing their theoretical efficiency advantages in practice.

\noindent\textbf{Current Solutions and Fundamental Limitations:}
The research community has responded to the MoE memory bottleneck with increasingly sophisticated optimization techniques~\cite{expertflow2024, promoe2024, fate2025, hobbit}, yet fundamental limitations persist. Traditional reactive approaches employ on-demand expert loading with LRU-based replacement policies~\cite{expertflow2024, promoe2024, fate2025, hobbit}, providing zero parallelism between expert loading and computation, while offering no predictive capabilities despite substantial structure in expert routing patterns.

Recent advances have introduced predictive prefetching strategies with promising but limited results. ExpertFlow~\cite{expertflow2024} achieves $2\times$ to $10\times$ speedup and up to 93.72\% GPU memory savings through token distribution optimization and adaptive caching, but effectiveness diminishes with more than 32 experts, far below the 128+ experts common in production models. ProMoE~\cite{promoe2024} demonstrates $2.20\times$ prefill and $2.07\times$ decode speedups through stride-based prefetching, but focuses on single-token optimization and edge deployment scenarios. Fate~\cite{fate2025} employs cross-layer gates for fast edge inference, achieving $4.5\times$ speedups on resource-constrained GPUs, but remains specialized for memory-limited environments. More recent systems like MoE-Gen~\cite{moegen2025} and Klotski~\cite{klotski2025} address batching optimization and expert-aware pipeline scheduling, but lack intelligent prediction capabilities.

These approaches face three systematic limitations that prevent comprehensive MoE optimization. First, existing systems optimize individual bottlenecks in isolation rather than addressing the coupled nature of memory bandwidth, computational efficiency, and prediction accuracy. Second, most solutions target single-request scenarios, missing substantial batch-level optimization opportunities where expert reuse patterns can provide exponential memory savings. Third, current predictive systems face a critical tradeoff between prediction accuracy and computational overhead, often requiring architecture-specific tuning that limits cross-model generalization.

\noindent\textbf{\design: Neural Prediction for Comprehensive MoE Optimization:}
This paper introduces \design, a comprehensive neural expert prefetching system that fundamentally transforms MoE inference from reactive memory management to predictive optimization. Our approach is motivated by a key insight: {\em expert routing in large language models exhibits rich spatio-temporal structure that can be systematically exploited through neural sequence modeling.} Through analysis of 37,200 routing traces totaling 3.34 million expert selections, we identify three fundamental forms of structure: temporal correlations providing 54\% uncertainty reduction, power-law popularity distributions ($\alpha = 1.293$) creating extreme concentration effects, and semantic clustering into 13 distinct expert specializations. These insights motivate our neural prediction architecture: a dense transformer model that employs cross-layer attention mechanisms to achieve 33.86\% expert prediction accuracy ($43\times$ improvement over random baseline) with only 0.32\% computational overhead.

\design operates as a plug-and-play solution that requires no modifications to pre-trained MoE models. The term ``plug-and-play'' refers specifically to deployment-time integration: while our neural predictor requires initial training on routing traces (typically 3-4 hours on a single GPU), this one-time cost is {\em amortized} across thousands of inference requests and avoids the prohibitively expensive process of retraining large MoE models. The system integrates neural prediction with batch-aware optimization exploiting sublinear expert scaling ($\mathbb{E}[U(B)] \sim B^{0.932}$) and hierarchical caching achieving 99.4\% hit rates, delivering $2\times$ to $15\times$ end-to-end speedup across diverse MoE architectures. Our work makes the following key {\bf contributions} that advance the state-of-the-art in MoE inference optimization:
%\begin{itemize}

    \noindent$\bullet$ \textbf{Theoretical Foundation for MoE Prediction:} First comprehensive analysis revealing ``spatio-temporal'' structure in expert routing across diverse MoE architectures through analysis of 37,200 routing traces, establishing information-theoretic bounds and validating power-law expert popularity scaling ($\alpha = 1.293$).
    
    \noindent$\bullet$ \textbf{Neural Inter-Layer Prediction Architecture:} Novel dense ``transformer predictor'' achieving 33.86\% expert prediction accuracy ($43\times$ over random baseline) with only 0.32\% computational overhead, employing cross-layer attention mechanisms to capture dependencies across layers. 
    
    \noindent$\bullet$ \textbf{Batch-Aware Expert Deduplication:} First systematic exploitation of sublinear expert scaling, %($\mathbb{E}[U(B)] \sim B^{0.932}$) 
    providing up to 87.6\% memory savings and 26.3\% bandwidth reduction through intelligent expert reuse in batch processing scenarios.
    
    \noindent$\bullet$ \textbf{Production-Ready Hierarchical Caching System:} Model-agnostic three-tier ``cache hierarchy'' with prediction-driven prefetching achieving 99.4\% hit rates under iso-cache constraints, demonstrating $2\times$ to $15\times$ end-to-end speedup across Switch Transformer and Qwen MoE architectures, respectively.
    
    \noindent$\bullet$ \textbf{Comprehensive Multi-Architecture Evaluation:} First systematic comparison between sparse (top-1) and dense (top-k) MoE routing with 770+ experimental configurations, including comparison against state-of-the-art systems (ExpertFlow, ProMoE) and theoretical validation of all claims.
%\end{itemize}

The remainder of this paper is organized as follows. \Cref{sec:background} provides background on MoE architectures and establishes the theoretical foundations for our approach. \Cref{sec:design} details the \design system design, including the neural predictor architecture, batch optimization algorithms, and hierarchical caching system. \Cref{sec:implementation} describes our implementation and deployment considerations and provides extensive experimental evaluation across multiple MoE architectures and comparison with state-of-the-art systems. \Cref{sec:related} discusses related work in MoE optimization and neural prediction. Finally, \Cref{sec:conclusion} concludes with implications for future research and deployment of MoE systems.

%%%%%%%%%% %%%%%%%%%% %%%%%%%%%% %%%%%%%%%% 